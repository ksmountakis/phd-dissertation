\section{Introduction}

A \emph{stochastic task network} is a directed acyclic graph $G(V, E)$ 
with each node in $V = \{1,\ldots,n\}$ representing a task with a random duration and 
each arc $(i,j)\in E$ representing a precedence-constraint between tasks $i$ and $j$,
specifying that task $j$ cannot start unless task $i$ has finished.
%
Such networks appear in several domains like project scheduling \cite{leus2011resource}, 
parallel computing \cite{shestak2008}, or even digital circuit design \cite{blaauw2008},
where there is a need to model a partial order of events with uncertain durations.
%
Postulating that a model of uncertainty is known, 
task durations are described by a random vector $D=(D_1,\ldots,D_n)$ with a known probability distribution.
%
In project scheduling, for example, the duration $D_i$ of task $i$ may turn out to be shorter or longer 
than a nominal value according to a certain distribution.
%In digital circuit design, a task $i$ corresponds to a gate with an input/output delay $D_i$ that 
%varies with a certain distribution across gates due to manufacturing imperfections.

A given task network is typically mapped to a \emph{realized schedule} (i.e. an assignment of start-times to tasks) via \emph{earliest-start dispatching};
i.e. observing outcome durations and starting a task immediately when precedence-constraints allow 
(i.e. not later than the maximum finish-time of its network predecessors).
Random durations make the realized start-time of a task (and the overall realized schedule makespan) also random.
Since \emph{PERT networks} \cite{malcolm1959}, 
a large body of literature focused on the problem of determining the makespan distribution \cite{adlakha1989},
eventually shown to be a hard problem \cite{hagstrom1990}. 
A variety of efficient heuristics have been developed so far (see \cite{blaauw2008}), 
among which Monte Carlo sampling remains, perhaps, the most practical.

Consider, for example, the stochastic task network in Fig.~\ref{fig-house}, 
detailing the plan of a house construction project,
assuming task durations are random variables that follow the uniform distribution within respective intervals.
With earliest-start dispatching, the overall duration of the project (i.e. the realized schedule makespan) 
will range between 12 and 20 days with an expected value of a little over 16 days.


\begin{figure}[!t]
	\centering
	\begin{subfigure}[b]{0.45\textwidth}
		\begin{tikzpicture}
			\begin{scope}[every node/.style={circle,thin,draw}]
 				\node (1) at (0,+0)	{\scriptsize 1};
				\node (2) at (1.2,-1)	{\scriptsize 2};
				\node (3) at (1.2,+1)	{\scriptsize 3};
				\node (4) at (2.4,-1)	{\scriptsize 4};
				\node (5) at (3.5,+1)	{\scriptsize 5};
				\node (6) at (2.4,-0)	{\scriptsize 6};
				\node (7) at (3.5,-1)	{\scriptsize 7};
				\node (8) at (4.5,+0)	{\scriptsize 8};
			\end{scope}

			\begin{scope}[>={Stealth[black]},
				every node/.style={fill=white,circle},
				every edge/.style={draw=black,thin}]
				\path [->] (1) edge (2);				
				\path [->] (1) edge (3);				
				\path [->] (2) edge (4);				
				\path [->] (3) edge (5);				
				\path [->] (3) edge (6);				
				\path [->] (4) edge (7);				
				\path [->] (5) edge (8);				
				\path [->] (6) edge (7);				
				\path [->] (7) edge (8);				
 			\end{scope}
		\end{tikzpicture}
		\subcaption{Example construction plan.}
		\label{fig-house-a}
	\end{subfigure}
	~
	\begin{subfigure}[b]{0.4\textwidth}
		\begin{center}
		    \begin{tabular}{ l l c }
			    & Tasks & Durations (days) \\ \hline
			    1. & Erect walls & 2-4  \\ 
			    2. & Finish walls & 3-5 \\ 
			    3. & Finish roof & 2-6 \\
			    4. & Install plumbing & 3-5 \\
			    5. & Finish exterior & 6-8 \\
			    6. & Install electricity & 3-5 \\
			    7. & Paint interior & 2-4 \\
			    8. & Finishing touches & 1-1 \\
		    \hline
		    \end{tabular}
		\end{center}
	\subcaption{Estimated task durations.}
	\label{fig-house-b}
	\end{subfigure}
	\caption{A motivating example.}
	\label{fig-house}
\end{figure}


This paper addresses a problem which, to our knowledge, has not been addressed in existing literature.
To motivate our problem, let us return to the earlier example and suppose task 7 
(``Paint interior'') is assigned to a painting crew charging \$100 per day.
Assume we are willing to hire them for at least 4 days (the maximum number of days they will need) and for at most 6 days; i.e. we have a budget of \$600 for painting.
With earliest-start dispatching, 7 may start within 8 to 15 days from the project start (the start-date of task 1).
A challenge that arises in this situation is deciding when to hire the painting crew,
because to allow for an expected makespan of a little over 16 days (as mentioned earlier),
we must book the painting crew from the 8-th day and until the 19-th day, at the excessive cost of \$1100.
% ... est requires an agile availability of resources that might be too costly
The solution we examine here, is to use a different dispatching strategy,
associating task 7 with a \emph{planned release-time}, $t_7$, 
before which it may not start even if installing plumbing and electricity are finished earlier than $t_7$.
If we choose that 7 may not start earlier than, e.g., $t_7 = 13$ days from the project start,
we only need to book the painting crew on the 13-th day until the 19-th day, for an acceptable cost of \$600.
However, the price to pay for this stability is an expected makespan increase to a little over 17 days.

Now suppose that after assessing our budget carefully it turns out that each task may deviate at most, say $w$ days, from its respective planned release-time.
The emerging question addressed in this paper is:
\begin{quote}
	Which planned release-times reach the desired level of stability\footnote{As in Bidot et al. \cite{bidot2009}, stability here refers to the extent that a predictive schedule (planned release-times in our case) is expected to remain close to the realized schedule.}
	while minimizing the incurred performance penalty?
\end{quote}

This problem does not involve resource-constraints.
However, task networks are often used in the area of resource-constrained scheduling under uncertainty (see \cite{beck2002,herroelen2005})
to represent solutions (e.g. the \emph{earliest-start policy} \cite{igelmund1983}, the \emph{partial-order schedule} \cite{policella2004,godard2005,bonfietti2014}).
Thus, our work is expected to be useful in dealing with associated problems,
such as distributing slack in a resource-feasible schedule to make it insensitive to durational variability \cite{davenport2014}.

\subsubsection*{Organization}
A formal problem statement and its LP formulation are presented in Section~2. 
As the resulting LP can be quite costly to solve,
Section~3 presents our main result, an efficient dynamic programming algorithm.
Section~4 concludes the paper and outlines issues to be addressed in future work.

