
\subsection{A motivating example}

Consider the task network in Fig.~\ref{fig-house} describing the plan of an example house construction project.
As in most (if not all) real-world projects, task durations are uncertain,
in our example we assume that each task duration will fall within a lower and an upper bound, according to the uniform distribution.
Using an earliest-start execution of the tasks,
the whole construction project will take between 12 and 20 days with an expected makespan of a little over 16 days.

%But given the uncertain nature of task durations, is this expected makespan actually achievable?
Suppose that task 7 (paint interior) is to be carried out by a painting crew that charge \$100 per day
and assume that we are willing to hire them for at least 4 days (the maximum number of days they will need) and for at most 6 days;
i.e. we have a maximum budget of \$600 for painting.
A consequence of following an earliest-start execution strategy is that
task 7  may start within 8 to 15 days from the begining date of the project.
A challenge that arises in this situation is deciding when to hire the painting crew.
To allow for the expected makespan of a little over 16 days we must hire the painting crew from the 8-th day and until the 19-th day, 
at the excessive cost of \$1100.
% ... est requires an agile availability of resources that might be too costly

Alternatively, suppose that we associate task 7 with a stable release-date $t_7$ before which it may not start,
even if the tasks of installing plumbing and electricity are finished earlier than $t_7$.
If we choose that task 7 may not start earlier than $t_7 = 15$ days from the project start,
we ensure that it will, in fact, start on the 15-th day, regardless of the realized durations of its predecessors.
This enables us to hire the painting crew on the 15-th day until the 19-th, at the cost of \$400.
However, our choice incurs a penalty on performance, as the expected makespan now becomes 18 days.
Since \$400 is below our available painting budget,
we may instead choose $t_7 = 13$ which means that we hire the painting crew on the 13-th day until the 19-th day, 
for a cost of \$600 and an expected makespan of a little over 17 days.

Suppose now that after assessing our budget carefully,
we have decided a maximum number of days $w_i$ that each task $i$ can deviate from its respective stable release-time $t_i$.
The arising problem addressed in this paper is:
given a stochastic task network, how can we determine stable release-times that 
enforce the desired stability constraints while minimizing the incurred expected makespan/tardiness increase?

%Choosing this release-date sufficiently far from the completion-dates of preceeding tasks 6 and 4,
%the start-date of task 7 will become more predictable, as in most cases it will be equal to $t_7$.
%Suppose that we decide task 7 cannot start earlier than $t_7 = 15$ days after the project starts.
%That is, we now follow a different earliest-start execution strategy which takes into account 
%not only precedence-constraints but also associated release-dates.

\begin{figure}[!t]
	\centering
	\begin{subfigure}[b]{0.45\textwidth}
		\begin{tikzpicture}
			\begin{scope}[every node/.style={circle,thin,draw}]
 				\node (1) at (0,+0)	{\scriptsize 1};
				\node (2) at (1.2,-1)	{\scriptsize 2};
				\node (3) at (1.2,+1)	{\scriptsize 3};
				\node (4) at (2.4,-1)	{\scriptsize 4};
				\node (5) at (3.5,+1)	{\scriptsize 5};
				\node (6) at (2.4,-0)	{\scriptsize 6};
				\node (7) at (3.5,-1)	{\scriptsize 7};
				\node (8) at (4.5,+0)	{\scriptsize 8};
			\end{scope}

			\begin{scope}[>={Stealth[black]},
				every node/.style={fill=white,circle},
				every edge/.style={draw=black,thin}]
				\path [->] (1) edge (2);				
				\path [->] (1) edge (3);				
				\path [->] (2) edge (4);				
				\path [->] (3) edge (5);				
				\path [->] (3) edge (6);				
				\path [->] (4) edge (7);				
				\path [->] (5) edge (8);				
				\path [->] (6) edge (7);				
				\path [->] (7) edge (8);				
 			\end{scope}
		\end{tikzpicture}
		\subcaption{Example construction plan.}
		\label{fig-house-a}
	\end{subfigure}
	~
	\begin{subfigure}[b]{0.4\textwidth}
		\begin{center}
		    \begin{tabular}{ l l c }
			    & Tasks & Durations (days) \\ \hline
			    1. & Erect walls & 2-4  \\ 
			    2. & Finish walls & 3-5 \\ 
			    3. & Finish roof & 2-6 \\
			    4. & Install plumbing & 3-5 \\
			    5. & Finish exterior & 6-8 \\
			    6. & Install electricity & 3-5 \\
			    7. & Paint interior & 2-4 \\
			    8. & Finishing touches & 1-1 \\
		    \hline
		    \end{tabular}
		\end{center}
	\subcaption{Estimated task durations.}
	\label{fig-house-b}
	\end{subfigure}
	\caption{Example task network representing a construction plan.}
	\label{fig-house}
\end{figure}

