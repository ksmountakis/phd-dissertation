	%%%%%%%%%%%%%%%%%%%%%%% file template.tex %%%%%%%%%%%%%%%%%%%%%%%%%
	%
	% This is a general template file for the LaTeX package SVJour3
	% for Springer journals.          Springer Heidelberg 2006/03/15
	%
	% Copy it to a new file with a new name and use it as the basis
	% for your article. Delete % signs as needed.
	%
	% This template includes a few options for different layouts and
	% content for various journals. Please consult a p revious issue of
	% your journal as needed.
	%
	%
	% 5th Mar 2012
	% NOTES for MISTA 2015
	% This template has been supplied for use for the MISTA 2015 conference
	% In essence it is the same as the one supplied by SV, but with additional comments
	% The reason this template is being used is to make it as easy as possible
	% to be able to submit to the post conference special issue of the Journal
	% of Scheduling
	%%%%%%%%%%%%%%%%%%%%%%%%%%%%%%%%%%%%%%%%%%%%%%%%%%%%%%%%%%%%%%%%%%%
	%
	% First comes an example EPS file -- just ignore it and
	% proceed on the \documentclass line
	% your LaTeX will extract the file if required
	\begin{filecontents*}{example.eps}
	%!PS-Adobe-3.0 EPSF-3.0
	%%BoundingBox: 19 19 221 221
	%%CreationDate: Mon Sep 29 1997
	%%Creator: programmed by hand (JK)
	%%EndComments
	gsave
	newpath
	  20 20 moveto
	  20 220 lineto
	  220 220 lineto
	  220 20 lineto
	closepath
	2 setlinewidth
	gsave
	  .4 setgray fill
	grestore
	stroke
	grestore
	\end{filecontents*}
	%
	% For MISTA 2015, use the default option that has been supplied
	\documentclass{svjour3}                     % onecolumn (standard format)
	%\documentclass[smallextended]{svjour3}     % onecolumn (second format)
	%\documentclass[twocolumn]{svjour3}         % twocolumn
	%
	\smartqed  % flush right qed marks, e.g. at end of proof
	%
	\usepackage{graphicx}
	%
	% \usepackage{mathptmx}      % use Times fonts if available on your TeX system
	%
	% insert here the call for the packages your document requires
 	\usepackage{latexsym}
	\usepackage{amsmath}
	\usepackage{amsfonts}
	\usepackage{graphicx}
 	\usepackage{subfig}
  	\usepackage{algorithmic}
  	\usepackage{algorithm}
  	\usepackage{enumerate}
  	\usepackage{microtype}
  	\usepackage{multirow}
 	
	% please place your own definitions here and don't use \def but
	% \newcommand{}{}
	
	\newcommand{\xsc}{\gamma}
	\newcommand{\xSC}{\Gamma}
	
	\newcommand{\xs}{\boldsymbol{s}}
 	\newcommand{\xd}{\boldsymbol{d}}
	\newcommand{\xq}{\boldsymbol{q}}
	\newcommand{\xb}{\boldsymbol{b}}
	\newcommand{\xl}{\boldsymbol{l}}
	\newcommand{\xf}{\boldsymbol{f}}
 	\newcommand{\xD}{\boldsymbol{D}}
  	\newcommand{\xt}{\boldsymbol{t}}
  	\newcommand{\xe}{\epsilon}
	
	\newcommand{\xE}{\mathcal{E}}
	\newcommand{\xS}{\boldsymbol{S}}
	\newcommand{\xX}{\boldsymbol{X}}
	\newcommand{\xP}{\boldsymbol{\Pi}}
	\newtheorem{edit}{Edit}	
	\newtheorem{TODO}{TODO}
	
	% Insert the name of "your journal" with
	% This is preset for MISTA 2015: Do not change
	\journalname{MISTA 2015}
	%
	\setcounter{tocdepth}{3}
	\begin{document}
	
	\title{Exact and heuristic methods for trading-off
	makespan and stability in stochastic project scheduling}
	
	\subtitle{}
	
	\author{Simon Mountakis \and
	        Tomas Klos \and \\
	        Cees Witteveen \and
	        Bob Huisman
	}

	\institute{Simon Mountakis \at
	              Delft University of Technology \\
	              \email{k.s.mountakis@tudelft.nl}  %  \\
	           \and
	           Tomas Klos \at
	              Delft University of Technology \\
	              \email{t.b.klos@tudelft.nl}  %  \\
	           \and
	           Cees Witteveen \at
	              Delft University of Technology \\
	              \email{c.witteveen@tudelft.nl}  %  \\
	           \and
	           Bob Huisman \at
	              NedTrain \\
	              \email{b.huisman@nedtrain.nl} %  \\
	}
	
	\maketitle
	
	\begin{abstract}
	
	This paper addresses a problem of practical value in project scheduling:
	trading expected makespan for stability,
	under stochastic activity duration uncertainty.
	We present the formal statement of a problem 
	that we name Proactive Stochastic RCPSP (PS-RCPSP).
	Assuming activity durations follow known probability distributions,
	PS-RCPSP asks to find a so-called earliest-start (es) policy and
	a proactive schedule that together minimize the weighted 
	sum of expected project makespan and expected instability
	(deviation of the realized from the proactive schedule).
	Extending an existing MILP model for the well-known deterministic 
	Resource-Constrained Project Scheduling Problem (RCPSP),
	we present a MILP model for PS-RCPSP, which allows us to 
	find optimal (es-policy, proactive schedule) pairs.
	To deal with instances of practical size,
	we propose a Linear Programming (LP)-based 
	and a Mixed-Integer LP (MILP)-based heuristic.
	Our LP-based heuristic optimizes the proactive schedule while 
	keeping the es-policy part of the solution fixed.
	Our MILP-based heuristic aims to optimize the structure of the policy together with the proactive schedule.
	In contrast to existing state-of-art approaches such as
	CCP \cite{lamas2014purely} and STC \cite{van2008proactive},
	our heuristics rely on optimizing the proactive schedule together with the scheduling policy.
 	Experiments show that the LP-based heuristic is efficient and compares favorably with the state-of-art 
	(i.e.\ achieves smaller expected makespan for a certain level of expected instability)
	when the aim is to achieve near-zero instability at the cost of higher makespan.
	The MILP-based heuristic seems more effective (albeit not as efficient) when the aim is to
	achieve low expected makespan at the cost of moderate or high instability.
 	\end{abstract}
	
%	\tableofcontents
	
	\section{Introduction}

A \emph{stochastic task network} is a directed acyclic graph $G(V, E)$ 
with each node in $V = \{1,\ldots,n\}$ representing a task with a random duration and 
each arc $(i,j)\in E$ representing a precedence-constraint between tasks $i$ and $j$,
specifying that task $j$ cannot start unless task $i$ has finished.
%
Such networks appear in several domains like project scheduling \cite{leus2011resource}, 
parallel computing \cite{shestak2008}, or even digital circuit design \cite{blaauw2008},
where there is a need to model a partial order of events with uncertain durations.
%
Postulating that a model of uncertainty is known, 
task durations are described by a random vector $D=(D_1,\ldots,D_n)$ with a known probability distribution.
%
In project scheduling, for example, the duration $D_i$ of task $i$ may turn out to be shorter or longer 
than a nominal value according to a certain distribution.
%In digital circuit design, a task $i$ corresponds to a gate with an input/output delay $D_i$ that 
%varies with a certain distribution across gates due to manufacturing imperfections.

A given task network is typically mapped to a \emph{realized schedule} (i.e. an assignment of start-times to tasks) via \emph{earliest-start dispatching};
i.e. observing outcome durations and starting a task immediately when precedence-constraints allow 
(i.e. not later than the maximum finish-time of its network predecessors).
Random durations make the realized start-time of a task (and the overall realized schedule makespan) also random.
Since \emph{PERT networks} \cite{malcolm1959}, 
a large body of literature focused on the problem of determining the makespan distribution \cite{adlakha1989},
eventually shown to be a hard problem \cite{hagstrom1990}. 
A variety of efficient heuristics have been developed so far (see \cite{blaauw2008}), 
among which Monte Carlo sampling remains, perhaps, the most practical.

Consider, for example, the stochastic task network in Fig.~\ref{fig-house}, 
detailing the plan of a house construction project,
assuming task durations are random variables that follow the uniform distribution within respective intervals.
With earliest-start dispatching, the overall duration of the project (i.e. the realized schedule makespan) 
will range between 12 and 20 days with an expected value of a little over 16 days.


\begin{figure}[!t]
	\centering
	\begin{subfigure}[b]{0.45\textwidth}
		\begin{tikzpicture}
			\begin{scope}[every node/.style={circle,thin,draw}]
 				\node (1) at (0,+0)	{\scriptsize 1};
				\node (2) at (1.2,-1)	{\scriptsize 2};
				\node (3) at (1.2,+1)	{\scriptsize 3};
				\node (4) at (2.4,-1)	{\scriptsize 4};
				\node (5) at (3.5,+1)	{\scriptsize 5};
				\node (6) at (2.4,-0)	{\scriptsize 6};
				\node (7) at (3.5,-1)	{\scriptsize 7};
				\node (8) at (4.5,+0)	{\scriptsize 8};
			\end{scope}

			\begin{scope}[>={Stealth[black]},
				every node/.style={fill=white,circle},
				every edge/.style={draw=black,thin}]
				\path [->] (1) edge (2);				
				\path [->] (1) edge (3);				
				\path [->] (2) edge (4);				
				\path [->] (3) edge (5);				
				\path [->] (3) edge (6);				
				\path [->] (4) edge (7);				
				\path [->] (5) edge (8);				
				\path [->] (6) edge (7);				
				\path [->] (7) edge (8);				
 			\end{scope}
		\end{tikzpicture}
		\subcaption{Example construction plan.}
		\label{fig-house-a}
	\end{subfigure}
	~
	\begin{subfigure}[b]{0.4\textwidth}
		\begin{center}
		    \begin{tabular}{ l l c }
			    & Tasks & Durations (days) \\ \hline
			    1. & Erect walls & 2-4  \\ 
			    2. & Finish walls & 3-5 \\ 
			    3. & Finish roof & 2-6 \\
			    4. & Install plumbing & 3-5 \\
			    5. & Finish exterior & 6-8 \\
			    6. & Install electricity & 3-5 \\
			    7. & Paint interior & 2-4 \\
			    8. & Finishing touches & 1-1 \\
		    \hline
		    \end{tabular}
		\end{center}
	\subcaption{Estimated task durations.}
	\label{fig-house-b}
	\end{subfigure}
	\caption{A motivating example.}
	\label{fig-house}
\end{figure}


This paper addresses a problem which, to our knowledge, has not been addressed in existing literature.
To motivate our problem, let us return to the earlier example and suppose task 7 
(``Paint interior'') is assigned to a painting crew charging \$100 per day.
Assume we are willing to hire them for at least 4 days (the maximum number of days they will need) and for at most 6 days; i.e. we have a budget of \$600 for painting.
With earliest-start dispatching, 7 may start within 8 to 15 days from the project start (the start-date of task 1).
A challenge that arises in this situation is deciding when to hire the painting crew,
because to allow for an expected makespan of a little over 16 days (as mentioned earlier),
we must book the painting crew from the 8-th day and until the 19-th day, at the excessive cost of \$1100.
% ... est requires an agile availability of resources that might be too costly
The solution we examine here, is to use a different dispatching strategy,
associating task 7 with a \emph{planned release-time}, $t_7$, 
before which it may not start even if installing plumbing and electricity are finished earlier than $t_7$.
If we choose that 7 may not start earlier than, e.g., $t_7 = 13$ days from the project start,
we only need to book the painting crew on the 13-th day until the 19-th day, for an acceptable cost of \$600.
However, the price to pay for this stability is an expected makespan increase to a little over 17 days.

Now suppose that after assessing our budget carefully it turns out that each task may deviate at most, say $w$ days, from its respective planned release-time.
The emerging question addressed in this paper is:
\begin{quote}
	Which planned release-times reach the desired level of stability\footnote{As in Bidot et al. \cite{bidot2009}, stability here refers to the extent that a predictive schedule (planned release-times in our case) is expected to remain close to the realized schedule.}
	while minimizing the incurred performance penalty?
\end{quote}

This problem does not involve resource-constraints.
However, task networks are often used in the area of resource-constrained scheduling under uncertainty (see \cite{beck2002,herroelen2005})
to represent solutions (e.g. the \emph{earliest-start policy} \cite{igelmund1983}, the \emph{partial-order schedule} \cite{policella2004,godard2005,bonfietti2014}).
Thus, our work is expected to be useful in dealing with associated problems,
such as distributing slack in a resource-feasible schedule to make it insensitive to durational variability \cite{davenport2014}.

\subsubsection*{Organization}
A formal problem statement and its LP formulation are presented in Section~2. 
As the resulting LP can be quite costly to solve,
Section~3 presents our main result, an efficient dynamic programming algorithm.
Section~4 concludes the paper and outlines issues to be addressed in future work.


	\section{Preliminaries}
	\label{sec-prelim}
 	
 	The purpose of this section is to introduce the research area of
 	proactive-reactive (stochastic) project scheduling,
 	which is where the contributions of this paper belong to.
  	To establish autonomy and to facilitate discussion in further sections,
  	we use convenient notation (which sometimes departs from standard notation)
  	and begin with summarizing existing concepts from 
  	deterministic and (purely) reactive project scheduling.
  	For a comprehensive survey of deterministic,
  	reactive and proactive-reactive project scheduling,
  	the reader may refer to \cite{herroelen2004robust}.
   	 	
 	\subsection{Deterministic project scheduling}
 	 	
	A project is usually represented as a directed acyclic graph $G(N,E)$,
	with nodes $N=\{1,\ldots,n\}$ corresponding to the set of $n$ project activities.
	Each directed arc $(i,j)$ in $E \subseteq \{(i,j) \in N^2\}$ defines a direct
	temporal constraint between activities $i$ and $j$,
	meaning that $j$ may not start unless activity $i$ has finished.
 	In effect, $E$ defines a binary, irreflexive and transitive relation:
 	if there is a path from activity $i$ to activity 
 	$j$ in $G(N,E)$ then $j$ cannot start unless $i$ has finished.
	Let us $T(E) \supseteq E$ denote the transitive closure of $E$, 
	defined as 
	\[
		T(E) := \{(i,j) \in N^2 : \exists 
		\textrm{ a path from } i \textrm{ to } j \textrm { in } G(N,E)\}
	\]
	We shall name a \emph{temporally independent set} each
	subset of activities $X \subseteq N$ which are mutually 
	independent with respect to temporal constraints.
	That is, if $X$ is a temporally independent set, then $X^2 \cap T(E) = \emptyset$.
	Obviously, if only temporal constraints are taken into account, 
	the activities of a temporally independent set may overlap in time in a schedule.
	
	We assume as input a set $R := \{1,\ldots,m\}$ of $m$ resources which must be shared among activities.
	Each resource $r \in R$ is associated with known capacity $b_r \in \mathbb{N}_0$.
	Furthermore, each activity $i \in N$ requires a known amount 
	$q_{ir} \leq b_r$ of resource $r$ while it executes.
	Vector $\xb \in \mathbb{N}^m_0$ and matrix $\xq \in \mathbb{N}^{n \times m}_0$
	define the problem's resource constraints.
	Every independent set $X$ for which $\sum_{i\in X} q_{ir} > b_r$ 
	for some $r\in R$ is called a \emph{forbidden} set.
	Even though it is allowed by the temporal constraints $E$, 
	all activities in $X$ may not overlap at some timepoint $t$ because
	resource $r$ will be used beyond its capacity, which is not possible.
	
	Let $H \subseteq N^2$ denote a set of temporal constraints.
	Below we give the definition of a function $\Phi$ which returns the set of all forbidden sets
	w.r.t. temporal constraints $H$ and the problem's resource constraints $(\xq,\xb)$.
	\begin{align}
		\Phi(H) := \{X \subseteq N: 
		X^2 \cap T(H)=\emptyset, 
		\sum_{i\in X} q_{ir} > b_r \textrm{ for some } r \in R\}
	\end{align}
	
	In addition to the parameters mentioned so far,
	we assume as input a vector $\xd \in \mathbb{N}^n_0$ such that $d_i$ defines the duration of activity $i$.
	Overall, a tuple $(N,R,E,\xd,\xq,\xb)$ specifies an instance of the RCPSP.
	A schedule $\xs \in \mathbb{N}^n_0$ 
	such that $s_i$ defines the start time of activity $i$,
	is a feasible solution when it satisfies the temporal and resource constraints,
	meaning that
	\begin{align}
		s_j \geq s_i + d_i \qquad & \forall (i,j) \in E \label{eq-precon} \\
		a(\xs, t) \notin \Phi(E)	\label{eq-rescon} & \qquad \forall t \geq 0
	\end{align}
 
	Here, $a(\xs,t) := \{i \in N : t \in [s_i, s_i+d_i)\}$ is the set of
	activities executing at timepoint $t$ according to $\xs$ and $\Phi$ as defined earlier.
	Thus, (\ref{eq-rescon}) ensures there is no timepoint $t$ at which the
	activities of a forbidden set overlap concurrently.
	
	\paragraph{Project source-sink convention.}
	RCPSP asks to find a feasible schedule of minimum makespan
	$C_{\max}(\xs) := \max \{ s_i + d_i : i \in N\}$.
	Most RCPSP-related works assume that the last activity, $n$, 
	is a dummy activity with zero duration (i.e. $d_n = 0$) and that
	it must wait for the completion of every other activity (i.e. $(i,n) \in T(E)$ for every $i \in N-\{n\}$).
	This dummy activity is often known as the project "sink" and it holds that $C_{\max}(\xs) = s_n$.
	Another convention of most RCPSP-related works is that the first activity, 1,
	often known as the project "source" must be waited on by every other activity
	(i.e. $(1,j) \in T(E)$ for every $j \in N-\{1\}$).
	
 	We shall hereafter assume activities 1 and $n$ correspond to the project source and sink, respectively.
 	The RCPSP can now be formally stated as:
	\begin{align}
		\xs^* := \arg \min \{s_n:  (\ref{eq-precon}), (\ref{eq-rescon}), \xs \geq 0 \}
	\end{align}
	
	\subsection{Reactive project scheduling}
 	
	In the research area of stochastic project scheduling,
	the activity durations vector $\xd$ is replaced with a stochastic vector $\xD$ such that $D_i$
	is the stochastic variable representing the uncertain duration of activity $i$,
	with a known probability distribution $\mathbb{P}[D_i = t]$.
	In line with recent works on S-RCPSP,
	we shall denote (elements of) stochastic vectors with a capital symbol.
			
	S-RCPSP is a purely reactive extension of RCPSP.
	The solution sought for is no longer a schedule, but a reactive scheduling policy. 
	A policy is a combinatorial object $\pi$ which parameterizes the mapping  from stochatic vector $\xD$
	to a corresponding realized schedule $\xS(\pi,\xD)$.
	Note that $\xS$ denotes a function which returns a vector of activity start times (of length $n$).
	Furthermore, if a realization $\xd$ of $\xD$ is passed as an argument, 
	then $\xS(\pi,\xd)$ denotes a deterministic vector.
	if $\xD$ is passed as an argument, $\xS(\pi,\xD)$ denotes a stochatic vector.
	
	Different classes of policies have been proposed in the literature
	\cite{mohring1984stochastic,mohring1985stochastic,stork2000branch,ashtiani2011new}.
 	One condition that all policy classes are expected to satisfy is that function
 	$\xS$ complies with the \emph{non-anticipativity constraint}:
	the decision to start activity $i$ at time $[\xS(\pi,\xD)]_i$ cannot rely on information from the feature:
	the value of $[\xS(\pi,\xD)]_i$ must be determined by time $t \leq [\xS(\pi,\xD)]_i$.
	Other features such as the structure of $\pi$ and 
	the definition of function $\xS$ depend on the class under study.
	
 	\paragraph{List-based policies.}
	
	Two classes of policies which are prominent in the literature
	are \emph{resource-based} (rb) policies and \emph{activity-based} (ab) policies,
	also known collectively as \emph{list-based policies}.
	A list-based policy is a priority vector $\xl \in \mathbb{R}^n$ assigning priority $l_i$ to activity $i$.
	Realized schedule $\xS(\xl, \xD)$ is computed by a 
	variant of the well-known parallel schedule-generation-scheme (SGS)
	complying with the non-anticipativity constraint \cite{ballestin2009resource};
	with the SGS definition being slightly different between rb-policies and ab-policies.
%	At $t=0$ and at each subsequent activity completion $t > 0$
%	one starts as many activities as allowed by temporal and resource constraints.
%	A similar class is that of \emph{activity-based} (ab) policies;
%	an ab-policy is again a priority vector.
%	But the behavior of $p$ is now slightly different:
%	a activity may not start at $t$ 
%	(even if temporal and resource constraints allow this) unless 
%	all activities of lower priority have finished by $t$.
%	Note that every realized schedule $\xS(\xl,\xD)$ is feasible, 
% 	since function $\xS$ forces the problem's precedence and resource constraints
% 	regardless of the choice of $\xl$.
 	As far as list-based policies are concerned, S-RCPSP asks to find a vector 
	$\xl \in \mathbb{R}^n$ that minimizes $\mathbb{E}[[\xS(\xl,\xD)]_n]$.
	%
	Stork \cite{stork2000branch} proposes exact branch-and-bound algorithms for both rb and ab-policies.
	Ballest{\'\i}n \cite{ballestin2007worthwhile} proposes
	an efficient genetic algorithm for ab-policies,
 	providing the first computational experience on larger S-RCPSP instances.
	Ballest{\'\i}n and Leus \cite{ballestin2009resource} manage 
	to obtain better results with a 
	Greedy Randomized Adaptive Search Procedure (GRASP), 
	again for the class of ab-policies.
	The best performance (w.r.t. expected makespan minimization)
	has so far been obtained with the more recent work of
	Ashtiani et al. \cite{ashtiani2011new} who propose a GRASP
	for a new class, namely \emph{pre-processing} (pp) policies--%
	a hybrid between rb-policies and es-policies.

 	\paragraph{Earliest-start policies.}
 	
   	An es-policy is a set of temporal constraints $\xE \subseteq N^2$ chosen such that
 	\begin{align}
 		T(\xE) \supseteq E, \label{es-1} \\
 		\Phi(\xE) = \emptyset \label{es-2}, \\
 		G(N,\xE) \textrm{ is acyclic} \label{es-3}
 	\end{align}
 	Condition (\ref{es-1}) ensures that a schedule $\xs$ satisfying
 	$s_j \geq s_i + d_i$ for each $(i,j)\in \xE$ 
 	(here $\xd$ can be any arbitrary choice of activity durations)
 	is feasible with respect to the problem's precedence constraints $E$.
  	Condition (\ref{es-2}) ensures that
  	$\xs$ satisfying $\xE$ implies that it also satisfies resource constraints 
  	prescribed by availabilities $\xb$ and requirements $\xq$ (as described earlier).
 	Condition (\ref{es-3}) ensures that the set of schedules satisfying $\xE$ 
 	(for any arbitrary choice of activity durations $\xd$) is non-empty.
 	
 	When a project is executed according to an es-policy $\xE$,
 	the schedule that is realized, $\xS(\xE,\xD)$, 
 	is what is often known as the \emph{earliest-start} schedule specified by $\xE$.
 	The earliest-start schedule of $\xE$ can be defined as:
 	\begin{align}
 		[\xS(\xE,\xD)]_j := \max \{ [\xS(\xE,\xD)]_i + D_i : (i,j) \in \xE\}
 	\end{align}
 	To put it simply, an activity $j$ starts immediatelly 
 	when all its predecessors in $G(N,\xE)$ have finished.
  	This time quantity (the latest finish time of $j$'s predecessors)
  	is often known as the length of the \emph{critical path} from project source $1$ to activity $j$.
  	As far as es-policies are concerned, the S-RCPSP asks to find some $\xE^*$ which
  	minimizes $[\xS(\xE,\xD)]_n$ (the length of the critical path to the project sink) by expectation:
 	\begin{align}
 		\xE^* := \arg \min \{\mathbb{E}[[\xS(\xE,\xD)]_n] : (\ref{es-1} - \ref{es-3}), \xE \in N^2\}
 		\label{eq-s-rcpsp}
 	\end{align}
  	Constraints $(\ref{es-1} - \ref{es-3})$ enforce the feasibility of any realized schedule
  	with both the precedence and the resource constraints of the problem.
  	
	Stork \cite{stork2000branch} proposed an exact branch-and-bound search for problem (\ref{eq-s-rcpsp}).
	His algorithm considers each \emph{minimal} forbidden set $X$
	(subset-minimal forbidden set) in some order and branches
	on each of $|\{(i,j) \in X^2\}|$ arcs which can be included in $\xE$
	in order to eliminate $X$ from $\Phi(\xE)$.
	Without obtaining new computational results,
	in \cite{leus2011resource} Leus gives a formal treatment of
	es-policies as resource-flow networks
	(flow networks which can represent feasible RCPSP schedules) and
	proposes a refined version of the branch \& bound algorithm of Stork.
	Exploiting the relation between resource-flows and es-policies,
	Artigues et al. \cite{leus2011robust} propose a 
	robust optimization model for es-policies,
	for when a stochastic characterization of uncertainty is not available.
	
\subsection{Proactive-reactive project scheduling}
	
	Reactive project scheduling allows one to pick activity start times dynamically
	during the project, under conditions of uncertainty.
	A main drawback of this approach 
	(e.g. \cite{herroelen2004construction,herroelen2004robust,braeckmans2005proactive})
	is that prior to (and during) project execution there is no 
	schedule prescribing activity start times that can more or less be trusted.
	Such a "proactive" schedule can serve important organizational purposes;
	in fact, the deviation of the realized schedule from this proactive schedule
	is expected to induce organizational costs.
	
	Attempts to overcome this drawback gave rise to the research area of
	proactive-reactive project scheduling,
	which is the research area that this paper belongs to.
	The main idea behind the proactive-reactive approach is to execute the project by using
	a proactive schedule together with a scheduling policy.
	Under uncertainty, some activities may not start at their proactive start times,
	because activities they have to wait for are not yet finished and/or
	resources they require are not yet released.
	In such cases, the scheduling policy determines which activities to start at their
	prescribed start times and which to postpone.
	It should be noted that most works assume \emph{railway-mode scheduling},
	meaning that an activity may not start earlier than its proactive start time,
	which strengthens the "stability" of the project execution.
	Clearly, the realized schedule is a function of the policy and the proactive schedule.	
	Achieving low instability (deviation of the realized from the proactive schedule)
	requires "spreading" proactive activity start times far appart, 
	in effect increasing the expected makespan.
	The general aim is to optimize some tradeoff between expected makespan and expected instability.
	
	Van de Vonder et al. \cite{van2006trade,van2008,van2007heuristic} propose and
	evaluate experimentally various proactive-reactive heuristics.
	The proposed heuristics assume as input an instance of S-RCPSP along with an initial schedule.
	The best performing heuristic is the so-called Starting Time Criticality (STC) heuristic.
	An es-policy is extracted from the structure of this initial schedule and used to
	iteratively transform the initial schedule into a 
	proactive schedule by inserting time-buffers betwen activities.
	%
	Deblaere et al. \cite{deblaere2011proactive} propose 
	an approach which integrates the proactive step (forming a proactive schedule) 
	and the reactive step (forming the adjoining policy).
	Their approach is only possible to compare with ours and others that assume railway-mode scheduling,
	by choosing sufficiently high penalties for earliness (w.r.t. the proactive schedule).%
	\footnote{We are grateful to one of our anonymous reviewers for this remark.}
 	%
	More recently, Vilches and Demeulemeester  \cite{lamas2015}
	propose a Chance-Constrained Programming model (CCP) for the RCPSP
	which asks to find a minimum makespan schedule
	subject to probabilistic temporal and resource constraints.
	They propose a Mixed Integer LP model, 
	the solution to which is a proactive schedule that will most likely remain 
	feasible under stochastic duration variability,
	without presumption on the policy that will be used during project execution.

		\section{Proactive Stochastic RCPSP}
	\label{sec-problem}
		
	In deterministic and reactive project scheduling,
	the main problem under study (RCPSP and S-RCPSP, respectively) is stated clearly.
	A clearly stated problem model cannot be found in proactive-reactive project scheduling literature,
	perhaps because this research area is still in a burn-in phase.%
	\footnote{Some works refer to \cite{herroelen2004construction} but this is a formal treatment of a
	proactive-reactive \emph{machine} scheduling problem.}%
	Existing literature seems to pursue the general aim 
	of optimizing some tradeoff between expected makespan and 
	expected instability (deviation from the proactive schedule).
	This section presents the formal statement of a proactive-reactive scheduling problem for which
	(heuristic and exact) solution methods are proposed in subsequent sections.
	
	The problem presented here, the Proactive Stochastic RCPSP (PS-RCPSP), 
	asks to find a tuple $(\xE,\xt)$ where $\xE$ is an es-policy and $\xt$ is a proactive schedule,
	minimizing the weighted sum of two performance criteria:
	\begin{enumerate}
		\item expected value of project makespan,
		\item expected value of tardiness with respect to proactive release-times.
	\end{enumerate}
	The first criterion measures lack of robustness and is relevant for obvious reasons.
	The second criterion measures instability and captures the expected 
	deviation of the realized schedule from the proactive schedule.
	Note that $\xt$ prescribes activity release-times (an activity $i$ may not start earlier than $t_i$).
	Intuitively, instability represents the extent to which the proactive start-times can be trusted,
	when used for organizational purposes before and during project execution.
 	
 	An instance of this problem is encoded by a tuple $(n,m,\xq,\xb,E,\xD,\alpha)$.
 	For clarity, we summarize the meaning of problem parameters.
 	Positive integer $n$ is the number of activities and $m$ is the number of resources.
 	Parameters $\xq \in \mathbb{N}^{m\times n}_0$ and $\xb \in \mathbb{N}^m_0$ 
 	define resource requirements and availabilities respectively.
 	Set $E \subseteq \{1,\ldots,n\}^2$ defines pairwise precedence constraints.
 	Stochastic vector $\xD$ is of length $n$ with each element 
 	$D_i$ a stochatic variable (with given distribution $\mathbb{P}[D_i = t]$)
 	which describes the uncertain duration of activity $i$.
 	Finally, parameter $\alpha \in [0,1]$ determines the desired tradeoff between
 	robustness (i.e. minimization of expected makespan) 
 	and stability (i.e. minimization of expected instability).
  	More emphasis can be put on either minimizining makespan (by choosing $\alpha$ closer to one)
  	or minimizing instability (by choosing $\alpha$ closer to zero).
	
 	Formally, the problem can be stated as follows:
 	\begin{align}
 		\min \qquad & f(\xE, \xt) := \alpha \cdot \mathbb{E} \left[ [\xS((\xE,\xt),\xD)]_n \right] + 
 									(1-\alpha) \cdot \mathbb{E} \left[ \sum_{i=1}^n 
 									([\xS((\xE,\xt),\xD)]_i - t_i) \right] 
 									\label{eq-psrcpsp-obj} \\
 		\textrm{s.t.} \qquad	&	\Phi(G(N,\xE)) = \emptyset \label{eq-psrcpsp-1} \\
 								&	T(\xE) \supseteq E \label{eq-psrcpsp-2} \\
 								&	G(N,\xE) \textrm{ is acyclic} \label{eq-psrcpsp-3} \\
 								&	\xE \in \{1,\ldots,n\}^2, \xt \geq 0
 	\end{align}
 	Conditions (\ref{eq-psrcpsp-2},\ref{eq-psrcpsp-3}) ensure there is a non-empty
 	set of schedules satisfying the problem's precedence constraints as prescribed in $E$.
 	Condition (\ref{eq-psrcpsp-1}) ensures that each such schedule also satisfies the
 	problem's resource constraints prescribed by $\xq$ and $\xb$.
 
	\section{Heuristic LP-based approach}
\label{sec-lp}
 	
 	This section presents a polynomial-time heuristic for PS-RCPSP which consists of two steps:
 	\begin{enumerate}
 		\item
 		using mean activity durations,
 		the deterministic RCPSP $(n,m,\xq,\xb,E,\mathbb{E}[\xD])$ is solved to obtain a good schedule $\xs$ and 
 		a feasible es-policy $\xE$ (i.e. satisfying (\ref{eq-psrcpsp-1}),(\ref{eq-psrcpsp-2}) and (\ref{eq-psrcpsp-3}))
 		is derived from the structure of $\xs$ in polynomial time 
 		(this procedure is described in \cite{artigues2003insertion}),
 		
 		\item
 		by solving a linear program presented below,
 		we find a proactive schedule $\xt$ that is optimally combined with $\xE$ (which is kept fixed)
 		so as to minimize an approximation of (\ref{eq-psrcpsp-obj}).
 	\end{enumerate}
  
 	After $\xE$ has been obtained in the first step,
 	finding $\xt$ which minimizes (an approximation of) the PS-RCPSP objective 
 	is achieved by solving the LP model presented below.	
 	
  	\begin{align}
 		\min \qquad & \left[ 
 			\alpha  \left( \frac{1}{|\xSC|} \sum_{\xsc \in \xSC} s^{\xsc}_n \right) +
 			(1-\alpha)  \left( \frac{1}{|\xSC|} \sum_{i=1}^{n}  \sum_{\xsc \in \xSC} (s^{\xsc}_i - t_i) \right)
  			\right] \label{eq-lp-obj} \\
 		\textrm{s.t.} \qquad & s^{\xsc}_j \geq s^{\xsc}_i + d^{\xsc}_i \qquad \forall (i,j) \in \xE, \xsc \in \xSC \\
 							 & s^{\xsc}_i \geq t_i \qquad \forall i=1,\ldots,n \\
 							 & \xt \geq 0
 	\end{align}
  	
 	Here, (\ref{eq-lp-obj}) approximates the objective (\ref{eq-psrcpsp-obj}) based on
 	$\Gamma \subseteq \mathbb{R}^n$:
    an adequately-sized sample of stochastic vector $\xD$.
  	The realization of activity durations under sample scenario $\xsc \in \Gamma$ 
 	is represented by vector $\xd^{\xsc} = (d^{\xsc}_1, \ldots, d^{\xsc}_n)$.
 	The corresponding realized schedule is 
 	$\textit{earliest-start}((\xE,\xt),\xd^{\xsc}) = (s^{\xsc}_1, \ldots, s^{\xsc}_n)$,
 	as computed by the model constraints.
 	The solution is a proactive schedule $\xt=(t_1,\ldots,t_n)$ that optimizes the tradeoff between
 	expected makespan and instability for the given es-policy $\xE$. 
 	This LP model has $n(|\Gamma|+1)$ linear variables 
 	($n$ variables $t_i$ and $n |\Gamma|$ variables $s^{\xsc}_i$).
 	

 \subsection{Related work}
\paragraph{Van de Vonder et al. \cite{van2008}} 

	propose several heuristics, 
  	of which the most competitive is the Starting Time Criticality (STC) 
 	heuristic and we shall therefore restrict our attention to it.
  	Our LP-based heuristic bears similarities with STC.
 	In fact, the first step of our heuristic is identical to that of STC:
 	an es-policy $\xE$ is extracted by the structure of an initial schedule $\xs$.
 	The second step of STC involves transforming the "unstable" schedule $\xs$
 	into a "stable" schedule $\xt$ with an iterative procedure, while keeping $\xE$ fixed.
 	In each iteration a one-unit time buffer is added at the start of that activity that "needs it the most" 
 	(as determined by a proposed "starting time criticality" measure)
 	until adding more buffer time would not further reduce the instability of $\xt$,
 	which is measured by
 	\begin{align}
 		\mathbb{E} \left[ \sum_{i=1}^n w_i ([\xS((\xE,\xt),\xD)]_i - t_i) \right] \label{eq-vonder-stb}
 	\end{align}
 	Here, $w_i$ is a cost associated with the instability of activity $i$.
 	Furthermore, $t_n$ is kept fixed to a project deadline and therefore $w_n$
 	represents the marginal cost of deviating from this project deadline.
 	Note that by replacing $\alpha$ in (\ref{eq-psrcpsp-obj}) 
 	with individual weights $w_i$ and choosing a fixed project deadline,
 	it is straightforward to adapt our approach to 
 	the instability objective considered by van de Vonder et al.
 	However, we felt that the choice of (\ref{eq-psrcpsp-obj}) as an objective is advantageous,
 	as it underlines the tradeoff between expected makespan and instability more clearly
 	and simplifies discussion by not involving a weight per individual activity and
 	not requiring the choice of a project deadline.
 	
	Note that $\xt$ is not guaranteed to be 
	(precedence and resource) feasible with respect to mean activity durations
	(as required in the work of van de Vonder \cite{van2008}).
	Enforcing $\xt$ to hold this property in our approach can be accomplished by
	including the following constraint in the LP model:
	\[
		t_j \geq t_i + \mathbb{E}[D_i] \qquad \forall (i,j) \in \xE
	\]
	However, this property only adds to the organizational value of $\xt$
	when mean values are reasonable estimates of activity durations.
	
	Let us note that both our heuristic and STC have 
	polynomial worst-case complexity (in the number of activities).
 	However, in contrast with STC, our approach guarantees that $\xt$ 
 	is chosen optimally when $\xE$ is kept fixed
 	and assuming the distribution of $\xD$ is approximated with a sample.
 	Therefore, if efficiency considerations enable us to choose a large-enough sample $\Gamma$
 	(which is mostly the case due to the efficiency of existing LP solvers),
 	our heuristic is expected to perform at least as well as STC.
 	Finally, note that our heuristic is simpler to implement,
 	requiring only the description of the presented LP model.

\paragraph{Leus et al. \cite{leus2004stability}}
assume as input a proactive schedule $\xt$ (e.g. one that has been produced by STC).
They propose a branch-and-bound search which returns the es-policy $\xE$ which fits $\xt$ optimally
in minimizing an expression of expected instability similar to (\ref{eq-vonder-stb}).
	\section{Exact MILP-based approach}
\label{sec-milp}
 	
 	PS-RCPSP (section~\ref{sec-problem}) asks to find an es-policy and a proactive schedule $(\xE,\xt)$
	that together minimize the weighted sum of expected makespan and instability.
	Section~\ref{sec-lp} presented a heuristic approach according to which 
	$\xE$ is kept fixed while $\xt$ is optimally paired with the policy by solving a LP.
	%
 	This section presents a Mixed Integer LP (MILP) model with which PS-RCPSP can be solved to optimality.
 	However, it should be pointed-out that a solution is trully exact only if we assume
 	stochastic duration distributions can be accurately described by the chosen sample $\Gamma$;
 	for general probability distributions we obtain a lower bound.
 	In fact, the problem of computing the exact expected makespan of a given es-policy
 	(and assuming duration distributions with discrete support) 
 	has been shown by Hagstrom in \cite{hagstrom1988computational} to be intractable.
 	However, our notion of exactness is in line with the computational study of Stork \cite{stork2000branch}
 	where "optimal" scheduling policies are computed by using a fixed sample of duration distributions.
 	
 	This model includes binary variables representing the 
 	structure of $\xE$ and linear variables representing $\xt$.
 	To our knowledge, no other exact approaches have been proposed in 
 	the literature for problems of similar type
 	(i.e. asking for a scheduling policy and proactive schedule that together optimize some
 	tradeoff between expected makespan and instability).
 	%In fact, and with the exception of \cite{deblaere2011proactive},
 	%existing heuristics for similar problems (e.g. \cite{van2008,lamas2015})
 	%do not attempt to optimize the scheduling policy and the proactive schedule together as a pair.
 	%
 	To arrive at this PS-RCPSP model, we merge the LP model presented in the previous section
 	with a MILP model that has been proposed by Artigues et al. \cite{artigues2003insertion}
 	and which allows to solve the deterministic RCPSP by treating it as a flow-network problem.
 	The model presented here is not entirely new,
 	since a similar technique (repeating precedence constraints for each
 	scenario of the chosen sample) has been proposed in \cite{leus2011robust}
 	for minimizing the maximum regret of an es-policy.
 		
	
\subsection{The RCPSP model of Artigues et al.}
	 
	Artigues et al. \cite{artigues2003insertion}
	represent a solution to the RCPSP as a so-called \emph{resource-flow}
	$\xf \in \mathbb{R}^{n\times n\times m}_0$;
	an assignment to variables $f_{ijr}$ associated 
	with each pair of activities $(i,j) \in N^2$ and each resource $r \in R$.
  	A resource-flow describes the "passing" of resource units inbetween activities.
 	More precisely, $\xf$ is an indirect representation of every schedule $\xs$ in which
 	$f_{ijr}$ units of resource $r$ are released by activity $i$ at its completion $s_i+d_i$
 	and then "picked up" by activity $j$ at its start $s_j$,
 	without another activity using these units between $s_i+d_i$ and $s_j$.
 	
 	A resource-flow is feasible when it satisfies
 	\begin{align}
 		\sum_{j \in N-\{i\}} f_{jir} = q_{ir}	\quad	\forall i \in N-\{1\} \label{eq-rf-1} \\
 		\sum_{j \in N-\{i\}} f_{ijr} = q_{ir} 	\quad	\forall i \in N-\{n\} \label{eq-rf-2}
  	\end{align}
 	Eq. $(\ref{eq-rf-1})$ asks that each activity $i$ (except for the sink)
 	receives as many resource units as it requires the moment it starts.
 	Eq. $(\ref{eq-rf-2})$ asks that each activity $i$ (except for the source)
	releases as many resource units as it has used the moment it finishes.
	
	The flow network $G(N,\phi(\xf))$ associated with $\xf$ is defined as
	$\phi(\xf) := \{(i,j) \in N^2: f_{ijr} > 0 \textrm{ for some } r \in R\}$;
	i.e. there is an arc from each activity to every other activity it
	passes at least one resource unit to.
  	As shown by Leus \cite{leus2011resource,leus2011robust},
  	feasible resource-flows and es-policies are interrelated:
  	$\xE = E \cup \phi(\xf)$ is a feasible es-policy if
  	$\xf$ is a feasible resource-flow (and $G(N,\xE)$ is acyclic).
  	Therefore, every schedule which satisfies $G(N,\xE)$ is feasible.
  	The following MILP model proposed by Artigues et al.
  	enables one to find a feasible resource-flow $\xf$
  	which minimizes the cost (described by function $g$)
  	of a schedule $\xs$ which satisfies the 
  	temporal constraints of $G(N,E \cup \phi(\xf))$.
  	
  	\begin{align}
  		\min \qquad			& s_n	\label{eq-art-obj}	\\
  		\textrm{s.t.} \qquad &	s_j \geq s_i + d_i - M(1-z_{ij}) \qquad \forall (i,j) \in N \label{eq-art-pre} \\
  			& z_{ij}=1 \qquad \forall (i,j) \in E \label{eq-art-zfix} \\
  			& f_{ijr} \leq M z_{ij} \qquad \forall (i,j) \in N^2, r\in R \label {eq-art-fz} \\
   			& (\ref{eq-rf-1}),(\ref{eq-rf-2}) \label{eq-art-inout} \\
  			& f_{ijr} \geq 0, z_{ij} \in \{0,1\} \qquad \forall (i,j) \in N^2, r\in R \label{eq-art-dom}
 	\end{align}
   	
%   	\begin{center}
%  	\begin{tabular}{ l l  r  l}
%  	min.	&	$s_n$									&	& \\
%  	s.t.	&	$f_{ijr} \leq M z_{ij}$					&	$\forall (i,j) \in N^2, r\in R$ & ($i$) \\
%  			&	$s_j \geq s_i + d_i - M(1-z_{ij})$		&	$\forall (i,j) \in N^2$	& ($ii$) \\
%  			&	$z_{ij} = 1$							&	$\forall (i,j) \in E$	& ($iii$) \\
%  			&	$f_{ijr} \geq 0, z_{ij} \in \{0,1\}$	&	$\forall (i,j) \in N^2, r\in R$ & ($iv$) \\
%  			&	(\ref{eq-rf-1}), (\ref{eq-rf-2})		& & ($v$)
%  	\end{tabular}
%  	\end{center}
  	
  	Here $M$ is a large constant.
  	Due to (\ref{eq-art-inout}) $\xf$ is a feasible resource-flow.
  	Due to (\ref{eq-art-fz}), if $f_{ijr} > 0$ for one or more $r \in R$ then $z_{ij} = 1$,
  	meaning that variables $z_{ij}$ describe the flow-network $\phi(\xf)$ of the resource-flow
  	(i.e. $\phi(\xf) = \{(i,j)\in N^2 : z_{ij}=1\}$).
  	Due to (\ref{eq-art-pre}) and (\ref{eq-art-zfix}), $\xs$ describes a schedule which 
  	satisfies the temporal constraints in $G(N,E\cup \phi(\xf))$.
  	Since $\xf$ is a feasible resource-flow, $\xs$ is a feasible schedule.
  	
\subsection{Extension for S-RCPSP}
  	
  	This section presents a trivial extension to the RCPSP model of Artigues et al.
  	which enables us to find optimal es-policies for the S-RCPSP.
   	Considering a sample $\Gamma \subset \mathbb{R}^n$ of stochastic activity durations vector $\xD$
   	allows us to present the following MILP model.
	
 	\begin{align}
 		\min \qquad 			& \frac{1}{|\xSC|} \sum_{\xsc \in \xSC} s^{\xsc}_n \label{eq-sart-obj} \\
 		\textrm{s.t.} \qquad	& s^{\xsc}_j \geq s^{\xsc}_i + d^{\xsc}_i - M(1-z_{ij}) 
 							   \qquad \forall (i,j) \in N^2, \xsc \in \xSC \label{eq-sart-pre} \\
 								& (\ref{eq-art-zfix} - \ref{eq-art-dom}) \label{eq-sart-dom}
 	\end{align}
   	
%  	\begin{center}
%  	\begin{tabular}{ l l  r  l}
%  	$\min$	&	$ \sum_{\gamma \in \Gamma}^n g(\xs^\gamma) $		&	& \\
%  	s.t.	&	$s^s_j \geq s^s_i + d^s_i - M(1-z_{ij})$	&	$\forall (i,j) \in N^2, s\in S$ & ($ii'$) \\
%  			&	(\ref{eq-rf-1}), (\ref{eq-rf-2}), ($i$), ($iii$), ($iv$)		& &	
%  	\end{tabular}
%  	\end{center}
  	
  	Our extension is rather straightfoward.
  	Each variable $s_i$ is included here as variable $s^{\xsc}_i$ for each sample scenario $\xsc \in \xSC$.
  	Precedence constraints (\ref{eq-art-pre}) from before are 
  	now replicated for each scenario $\xsc \in \xSC$ in condition (\ref{eq-sart-pre}).
  	Objective (\ref{eq-art-obj}) is now replaced with objective (\ref{eq-sart-obj}),
   	which estimates the makespan expectation $\mathbb{E}[S(E\cup \phi(\xf))]$ based on sample $\Gamma$.

\subsection{Extension for PS-RCPSP}
  	
  	Here we extend the previous model by including a variable $t_i$ for each $i\in N$,
  	which determines the activity's proactive starting time.
  	The resulting PS-RCPSP MILP model is presented below.
  	
  	\begin{align}
 		\min \qquad & \left [
 			\alpha  \left( \frac{1}{|\xSC|} \sum_{\xsc \in \xSC} s^{\xsc}_n \right) +
 			(1-\alpha)  \left( \frac{1}{|\xSC|} \sum_{i=1}^{n}  \sum_{\xsc \in \xSC} (s^{\xsc}_i - t_i) \right)
  			\right] \label{eq-milp-obj} \\
 		\textrm{s.t.} \qquad	& s^{\xsc}_j \geq s^{\xsc}_i + d^{\xsc}_i - M(1-z_{ij}) 
 							   \qquad \forall (i,j) \in N^2, \xsc \in \xSC \label{eq-milp-pre} \\
 								& (\ref{eq-art-zfix} - \ref{eq-art-dom}) \label{eq-milp-dom} \\
 								& s^{\xsc}_i \geq t_i \qquad i \in N \label{eq-milp-pro}, \xsc \in \xSC \\
 								& \xt \geq 0
 	\end{align} 	
 	
   	The objective now becomes identical to that of the LP-based heuristic,
  	measuring the weighted sum of expected makespan and expected instability.
  	Condition (\ref{eq-milp-pro}) ensures that an activity 
  	may not start earlier than its proactive start time.
  	
  	To summarize, by solving this model we obtain a PS-RCPSP solution $(\xE,\xt)$
  	where $\xE = \{(i,j) \in N^2: z_{ij} = 1\}$ is a feasible es-policy and $\xt$ defines a proactive schedule.
  	
 
 	\begin{proposition}
 	\label{pro-1}

 	Define $\xE := \{(i,j) : z_{ij} = 1\}$ the arcs of the flow-network
 	$G(N, E \cup \phi(\xf))$ associated with resource-flow $\xf$.
 	Let $\xf, \boldsymbol{z}, \xs, \xt$ be an optimal solution.
 	For each scenario $\xsc \in \xSC$,
 	$\xs^{\xsc}$ defines a schedule where each activity $i$ starts as soon as
 	allowed by its proactive release-time $t_i$ and es-policy $\xE$.
  	\end{proposition}
 	
 	\begin{proof}
 	For each scenario $\xsc \in \xSC$,
 	let $\bar{x}^{\xsc}$ denote the earliest allowed start time for activity $i$,
 	allowed by the combination of $\xE$ and proactive schedule $\xt$.
 	We want to prove that in an optimal solution, 
 	$\xs^{\xsc} = \bar{x}^{\xsc}$ for all $\xsc \in \xSC$.
 	
 	Assume that $s^{\xsc}_i = \bar{x}^{\xsc}_i + \delta$ for some $\xsc \in \xSC, i \in N$, with $\delta > 0$.
 	Since (\ref{eq-milp-obj}) increases monotonically with $s^{\xsc}_i$,
 	the objective can be improved by setting $s^{\xsc}_i = \bar{x}^{\xsc}_i$,
 	without violating any constraints.
 	Therefore, in every optimal solution we have $s^{\xsc}_i = \bar{x}^{\xsc}_i$ 
 	for all $\xsc \in \Gamma$ and $i \in N$,
 	meaning that each $\xs^{\xsc}$ defines the earliest start times schedule
 	allowed by the combination of by es-policy $\xE$ and proactive schedule $\xt$ under scenario $\xsc$.
 	\hfill $\Box$
 	\end{proof}
	
	By proposition~\ref{pro-1} it follows that an optimal solution to the MILP model
	presented above is, in fact, an optimal solution for the PS-RCPSP.


%\subsection{Related work}
%\paragraph{Braeckmans et al. \cite{braeckmans2005proactive}}
%\paragraph{Leus et al. \cite{leus2011robust}}

		\section{Heuristic MILP-based approach}
 	\label{sec-milp-heuristic}
  
 	
 	Even for small instances (e.g. with 30 activities and 4 resources),
 	solving the proposed model might take an inordinate amount of time.
 	We propose an algorithm (Algorithm~\ref{alg-milp-heuristic}),
 	the main idea of which was inspired by the \emph{iterative flattening}
 	heuristic of Oddi et al. \cite{oddi2009iterative}.
 	The heuristic of Oddi et al. was developed for the
 	deterministic RCPSP with minimum/maximum time-lag precedence constraints \cite{herroelen2001note}.
 	Every feasible schedule for the problem they study is compactly represented 
 	as a network of temporal constraints (known as a Simple Temporal Network \cite{dechter1991}).
 	It is the similarity with an es-policy (which is a network of zero-lag temporal constraints)
 	that has inspired the development of the heuristic presented here. 
  
 	The proposed heuristic involves solving a sequence of 
 	sufficiently small subproblems with non-increasing optimal objective values.
 	Each iteration involves solving a partially solved instance to optimality.
 	Thus, worst-case complexity is exponential in the number of activities.
 	In practice, however, "good" solutions can be obtained with relative efficiency.
  	
 	\begin{algorithm}
 		\caption{Iterative flattening for PS-RCPSP}
 		\label{alg-milp-heuristic}
 		\begin{algorithmic}[1]
 		\State $\xs \leftarrow$ schedule for RCPSP $(N,R,E,\mathbb{E}[\xD],\xq,\xb)$
 		\State $\xE^{*} \leftarrow E \cup \phi(\xf^{\xs})$ with $\xf^{\xs}$ extracted from $\xs$
 		\State $\xt^{*} \leftarrow (0,\ldots,0)$
 		%\State $k \leftarrow 0$
 		
 		\While {termination criteria not met}
 			\State $\mathcal{H} \leftarrow$ random subset of $\xE^{*}-T(E)$ chosen by criticality probability
 			\State $(\xE,\xt) \leftarrow$ 
 					optimal solution for PS-RCPSP $(N,R,\xE^{*}-\mathcal{H},\xD,\xq,\xb,\alpha)$
 			\If {$(\xE,\xt)$ has a lower objective than $(\xE^{*},\xt^{*})$}
 				\State $(\xE^{*},\xt^{*}) \leftarrow (\xE,\xt)$
  			\EndIf
  			%\State $k \leftarrow k + 1$
 		\EndWhile
 		\State return $(\xE^{*},\xt^{*})$
 		\end{algorithmic}
 	\end{algorithm}	
 	
 	Algorithm~\ref{alg-milp-heuristic} assumes as input an instance of the
 	PS-RCPSP.
 	According to aforementioned notation,
 	the instance is represented as $(N,R,E,\xD,\xq,\xb,\alpha)$.
 	An initial solution is obtained by solving a
 	deterministic RCPSP (lines 1-3)
 	which can be done efficiently with one of the various existing heuristics.
 	This solution will serve as a starting point for the first iteration,
 	which is described as follows.
 	A partial solution is formed by removing a random 
 	subset of highly critical arcs from the current solution (line 6).
 	The resulting subproblem is solved to optimality (by use of the proposed model)
 	and a complete solution is obtained (line 7).
 	If this new solution is better,
 	it becomes the starting point of the next iteration.
 	The algorithm may terminate when, e.g., a chosen
 	number of iterations have been performed,
 	or the objective has failed to improve a certain number of times. 
 	
 	\paragraph{Further efficiency improvements.}
 	Note that the optimal solution $(\xE,\xt)$   
 	of the subproblem solved in each iteration
 	cannot be worse than the best solution seen so far, $(\xE^*,\xt^*)$.
 	To improve performance one may use $(\xE^*,\xt^*)$ as an initial solution when solving the model (line 6).
 	Efficiency can be further improved by reducing the number of binary variables $z_{ij}$ in the model.
 	This can be accomplished by observing that
 	$z_{ij}$ for each $(i,j) \in T(\xE^* -  \mathcal{H})$ can be fixed to one and
 	$z_{ij}$ for each $(j,i) \in  T(\xE^* - \mathcal{H})$ can be fixed to zero.

 	\section{Experiments}
\begin{gap}
\ldots
\end{gap}
 	\section{Conclusions and Discussion}

%Given a stochastic task network with $n$ tasks we examined dispatching the tasks as early as possible but not earlier than respective planned release-times.
%Assuming a sample with $m$ realizations of the random task durations vector is given,
%we examined the problem of finding planned release-times that minimize the performance penalty of reaching a desired level of stability.

Given a stochastic task network with $n$ tasks we consider dispatching the tasks as early as possible, subject to (planned) release-times.
Assuming a sample with $m$ realizations of the stochastic durations vector is drawn, we defined an LP for finding 
optimal release-times; i.e. that minimize the performance penalty of reaching a desired level of stability.
%which can be solved at an estimated cost of $O(n^5 m^4)$
%Modern LP solvers can find the optimal with an estimated complexity of $O(n^5 m^4)$, which might be daunting for instances of practical size.
The resulting LP is costly to solve, so
pursuing a more efficient solution method we managed to show that optimal release-times can be expressed 
as a function of the earliest start time solution of an associated Simple Temporal Problem. 
Exploiting the structure of this STP, we were able to define a dynamic programming algorithm 
for finding optimal release-times with considerable efficiency, in time $O(n^2 m)$.

\subsubsection*{Future work}
Since we optimize according to a manageable sample $\z{P}$,
there is a (potentially non-zero) probability $\mathbb{P}_v$ that the realized start-time 
of a task deviates further than $w$ time-units from its planned release-time.
The question of how $\mathbb{P}_v$ (or $\mathbb{E}[\mathbb{P}_v]$ as in \cite{calafiore2005})
depends on $m$ (the size of $\z{P}$) should be addressed in future work.
%
Furthermore, in an earlier paper \cite{mountakis2015}, an LP similar to $(P)$ was used it in a two-step heuristic
for a flavor of the \emph{stochastic resource constrained project scheduling problem} (stochastic RCPSP) \cite{van2008,lamas2015}.
Given a resource allocation determined in a first step, in a second step a LP was used to find planned release-times 
that minimize the total expected deviation of the realized schedule from those release-times.
This heuristic was found to outperform the state-of-the-art in the area of \emph{proactive project scheduling}.
In future work, we shall investigate using the algorithm presented here in order to stabilize the given resource-allocation,
expecting gains in both efficiency and effectiveness.
%
Finally, a potentially related problem, namely PERTCONVG, is studied by Chr{\'e}tienne and Sourd in \cite{chretienne2003},
which involves finding start-times for a task network so as to minimize the sum of convex cost functions.
In fact, their algorithm bears structural similiarities to ours, since subproblems are solved in a topological order.
It would be worth investigating if their analysis can be extended in order to enable
casting the problem studied here as an instance of that problem.

 	
	\begin{acknowledgements}
	This research belongs to the \emph{Job Scheduling Problem} part of the 
	\emph{Rolling Stock Life Cycle Logistics} applied research and development programme, 
	funded by \emph{NS/NedTrain}.
	\end{acknowledgements}
	
	% BibTeX users please use one of
	\bibliographystyle{plain}      % basic style, author-year citations
	%\bibliographystyle{spmpsci}      % mathematics and physical sciences
	%\bibliographystyle{spphys}       % APS-like style for physics
	\bibliography{p}   % name your BibTeX data base
	
	% Non-BibTeX users please use
	%\begin{thebibliography}{}
	%
	% and use \bibitem to create references. Consult the Instructions
	% for authors for reference list style.
	%
	%\bibitem{RefJ}
	% Format for Journal Reference
	%Author, Article title, Journal, Volume, page numbers (year)
	% Format for books
	%\bibitem{RefB}
	%Author, Book title, page numbers. Publisher, place (year)
	% etc
	%\end{thebibliography}
	
	\end{document}
	% end of file sample.tex
	
	