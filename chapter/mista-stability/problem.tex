	\section{Proactive Stochastic RCPSP}
	\label{sec-problem}
		
	In deterministic and reactive project scheduling,
	the main problem under study (RCPSP and S-RCPSP, respectively) is stated clearly.
	A clearly stated problem model cannot be found in proactive-reactive project scheduling literature,
	perhaps because this research area is still in a burn-in phase.%
	\footnote{Some works refer to \cite{herroelen2004construction} but this is a formal treatment of a
	proactive-reactive \emph{machine} scheduling problem.}%
	Existing literature seems to pursue the general aim 
	of optimizing some tradeoff between expected makespan and 
	expected instability (deviation from the proactive schedule).
	This section presents the formal statement of a proactive-reactive scheduling problem for which
	(heuristic and exact) solution methods are proposed in subsequent sections.
	
	The problem presented here, the Proactive Stochastic RCPSP (PS-RCPSP), 
	asks to find a tuple $(\xE,\xt)$ where $\xE$ is an es-policy and $\xt$ is a proactive schedule,
	minimizing the weighted sum of two performance criteria:
	\begin{enumerate}
		\item expected value of project makespan,
		\item expected value of tardiness with respect to proactive release-times.
	\end{enumerate}
	The first criterion measures lack of robustness and is relevant for obvious reasons.
	The second criterion measures instability and captures the expected 
	deviation of the realized schedule from the proactive schedule.
	Note that $\xt$ prescribes activity release-times (an activity $i$ may not start earlier than $t_i$).
	Intuitively, instability represents the extent to which the proactive start-times can be trusted,
	when used for organizational purposes before and during project execution.
 	
 	An instance of this problem is encoded by a tuple $(n,m,\xq,\xb,E,\xD,\alpha)$.
 	For clarity, we summarize the meaning of problem parameters.
 	Positive integer $n$ is the number of activities and $m$ is the number of resources.
 	Parameters $\xq \in \mathbb{N}^{m\times n}_0$ and $\xb \in \mathbb{N}^m_0$ 
 	define resource requirements and availabilities respectively.
 	Set $E \subseteq \{1,\ldots,n\}^2$ defines pairwise precedence constraints.
 	Stochastic vector $\xD$ is of length $n$ with each element 
 	$D_i$ a stochatic variable (with given distribution $\mathbb{P}[D_i = t]$)
 	which describes the uncertain duration of activity $i$.
 	Finally, parameter $\alpha \in [0,1]$ determines the desired tradeoff between
 	robustness (i.e. minimization of expected makespan) 
 	and stability (i.e. minimization of expected instability).
  	More emphasis can be put on either minimizining makespan (by choosing $\alpha$ closer to one)
  	or minimizing instability (by choosing $\alpha$ closer to zero).
	
 	Formally, the problem can be stated as follows:
 	\begin{align}
 		\min \qquad & f(\xE, \xt) := \alpha \cdot \mathbb{E} \left[ [\xS((\xE,\xt),\xD)]_n \right] + 
 									(1-\alpha) \cdot \mathbb{E} \left[ \sum_{i=1}^n 
 									([\xS((\xE,\xt),\xD)]_i - t_i) \right] 
 									\label{eq-psrcpsp-obj} \\
 		\textrm{s.t.} \qquad	&	\Phi(G(N,\xE)) = \emptyset \label{eq-psrcpsp-1} \\
 								&	T(\xE) \supseteq E \label{eq-psrcpsp-2} \\
 								&	G(N,\xE) \textrm{ is acyclic} \label{eq-psrcpsp-3} \\
 								&	\xE \in \{1,\ldots,n\}^2, \xt \geq 0
 	\end{align}
 	Conditions (\ref{eq-psrcpsp-2},\ref{eq-psrcpsp-3}) ensure there is a non-empty
 	set of schedules satisfying the problem's precedence constraints as prescribed in $E$.
 	Condition (\ref{eq-psrcpsp-1}) ensures that each such schedule also satisfies the
 	problem's resource constraints prescribed by $\xq$ and $\xb$.
