\section{Introduction}
	
	% -- deterministic RCPSP
	
	Project scheduling literature mostly concentrates on scheduling
	subject to temporal and resource constraints.
	The schedule sought for is an assignment of start-times to activities,
	facilitating the efficient use of limited resources
	in order to minimize a lateness measure such as the project makespan.
	Finding a schedule usually invovles solving
	the Resource-Constrained Project Scheduling Problem (RCPSP) 
	(see \cite{artigues2013resource,hartmann2010survey} for comprehensive surveys).
	This problem has been shown to be \textsc{NP}-Hard \cite{blazewicz1983scheduling} and
	finds several industrial applications (e.g. \cite{bomsdorf2008model,bartels2009scheduling}).
	Associated literature includes numerous exact and (meta-)heuristic algorithms,
	able to find good schedules for large instances and diverse lateness measures
	(e.g. \cite{schutt2013solving,kone2011event,kolisch2006experimental,de2014novel}).
	%
	% -- transition from RCPSP to S-RCPSP because of uncertainty
	%
	Solving an RCPSP serves the purpose of preparing a feasible schedule,
	assuming a static deterministic project execution environment.
 	In practice, however, this assumption is rarely valid.
 	Activity durations used for preparing the schedule are mostly rough estimates,
 	since most projects are subject to delays during execution and
 	the final realized schedule is the result of 
 	subjecting the original schedule to modifications
 	which make it consistent with the project constraints in the face of delays.
 	Ad-hoc modifications lead to realized activity start-times 
 	that might differ from planned start-times,
  	compromising project predictability and timeliness.
	
	This paper addresses the issue of hedging against project uncertainty 
 	by preparing a schedule in combination with an execution strategy for coping with delays.
	In line with other works in \emph{stochastic project scheduling} 
	(see \cite{herroelen2004robust} for a comprehensive review)
	we assume activity durations are given as stochastic variables with known distributions
	and propose a new \emph{proactive-reactive} scheduling method.
	First, we define the Proactive Stochastic (PS) RCPSP as an extension of RCPSP.
	The solution to PS-RCPSP is a \emph{proactive schedule} and an \emph{earliest-start (es-) policy}
	that together minimize the weighted sum of the realized schedule's expected makespan 
	and its expected deviation from the proactive schedule.
	%
	PS-RCPSP is, in fact, a generalization of the Stochastic RCPSP (S-RCPSP),
	which asks to find a \emph{stochastic scheduling policy} instead of a schedule
	and various \emph{classes} of policies have been proposed in the literature 
	\cite{mohring1984stochastic,mohring1985stochastic,herroelen2004robust}.
	In general, a  policy defines a mapping between activity duration realizations to realized schedules.
	S-RCPSP asks to find a policy that minimizes the expected project makespan,
	with only few exact and heuristic approaches (mainly meta-heuristics) proposed over the last decade
	\cite{stork2000branch,ballestin2007worthwhile,ballestin2009resource,ashtiani2011new}.	
	
	Not preparing the project execution based on a schedule that can more or less be trusted
	(but rather, letting the realized schedule unfold during execution) 
	has been recognized as a shortcoming of S-RCPSP \cite{herroelen2004construction}.
	This shortcoming motivates us and a number of 
	other authors to propose a proactive-reactive approach,
	with \cite{van2008,deblaere2011proactive,lamas2015}
	yielding the most promising computational results in existing literature.
	Different approaches pursue different optimization objectives;
	however, the common aim is to optimize some \emph{tradeoff} between expected makespan 
	and expected deviation from the proactive schedule.
 	In line with other authors we represent activity duration distributions with a sample
 	and propose a Linear Programming (LP)-based heuristic for PS-RCPSP,
 	a Mixed-Integer LP (MILP) model enabling us to obtain exact solutions, and
 	a MILP-based heuristic which asimilates \emph{iterative flattening} \cite{oddi2009iterative}.
   	The (MI)LP models for PS-RCPSP proposed in this paper are the 
 	result of adjusting the models proposed by Artigues et al. in
 	\cite{artigues2003insertion} and \cite{leus2011robust}.	
 	We refer to exact solutions assuming stochastic duration distributions 
 	can be represented exactly by a sufficiently large sample.\footnote{
 	This notion of exactness is in line with Stork \cite{stork2000branch} who 
 	represent stochastic duration distributions with a sample
 	when proposing exact search methods for the S-RCPSP.}
 	
 	
%%%	Following the definition of PS-RCPSP we propose a heuristic based on Linear Programming (LP),
%%%	a Mixed Integer LP (MILP) model for PS-RCPSP that enables the finding exact solutions, and
%%%	a MILP-based heuristic that involves solving a sequence of smaller subproblems.
%%%	Similar proactive-reactive scheduling methods have been proposed by a number of authors,
%%%	with the heuristics in \cite{van2008,deblaere2011proactive,lamas2015} 
%%%	giving the most promising computational results.
%%%	Experiments with the known PSPLIB benchset \cite{kolisch1997psplib}
%%%	show that our heuristics compare favorably with prior art.
%%%	
%%%	This paper addresses the issue of hedging against project uncertainty 
%%% 	by preparing a schedule in combination with an execution strategy for coping with delays and
%%%	assuming activity durations are stochastic variables with known probability distributions.
%%% 	We propose a Mixed Integer Linear Programming (MILP) model along with two associated heuristics
%%% 	for a ploblem that we name the Proactive Stochastic RCPSP (PS-RCPSP).
%%% 	PS-RCPSP is, in fact, a generalization of an extension to the RCPSP,
%%% 	known as the Stochastic RCPSP (S-RCPSP),
%%% 	which is the main problem of study in the area of stochastic project scheduling \cite{herroelen2004robust}
%%%	(also known as reactive project scheduling).
%%%	S-RCPSP asks to find a \emph{stochastic scheduling policy} instead of a schedule.
%%%	Various \emph{classes} of policies have been proposed in the literature 
%%%	\cite{mohring1984stochastic,mohring1985stochastic,herroelen2004robust} but in general,
%%%	a policy is a combinatorial object representing a set of rules
%%%	that may guide the materialization of a final realized schedule,
%%%	as actual activity durations become known during project execution.
%%%	In effect, the policy defines a mapping from
%%%	every possible realization of activity durations to a corresponding realized schedule.
%%%	S-RCPSP asks to find a policy that minimizes the expected value of some project lateness measure
%%%	(e.g. the expected project makespan).
%%%	Few exact and heuristic S-RCPSP solution algorithms have been developed over the last decade
%%%	\cite{stork2000branch,ballestin2007worthwhile,ballestin2009resource,ashtiani2011new}.
%%%	
%%%	% -- transition from S-RCPSP to PS-RCPSP
%%%	
%%%	A main characteristic of S-RCPSP is that no schedule is prescribed prior to project execution.
%%%	Activity start-times are determined in a purely dynamic manner by the scheduling policy.
%%%	This has been recognized as a drawback (e.g. \cite{herroelen2004construction}) 
%%%	since the existence of a \emph{proactive}
%%%	schedule that can more or less be trusted may serve important organizational purposes.
%%%	An attempt to overcome this drawback gave rise to the research area of proactive-reactive project scheduling
%%%	which  is the research area that our work belongs to.
%%% 	The main idea behind the proactive-reactive approach is to execute the project by using
%%%	a proactive schedule together with a scheduling policy.
%%% 	The general aim of prominent approaches in this area 
%%%  	(e.g. \cite{van2008,deblaere2011proactive,lamas2015}) is to
%%% 	optimize some tradeoff between expected makespan and 
%%%	expected instability (deviation of the realized from the proactive schedule).
%%%		
%%%	The problem studied here, the PS-RCPSP, asks to find
%%%	a \emph{proactive} schedule paired with an \emph{earliest-start} (es) policy.
%%%	An es-policy is a set of temporal constraints between pairs of activities.
%%%	The proactive schedule prescribes release-times 
%%%	(i.e. earliest possible start-times) for activities.
%%%	An activity may start at its prescribed release-time or later in case of delays
%%%	(i.e. other activities requiring more than allocated time) as determined by the es-policy.
%%%	In effect, the tuple (proactive schedule, es-policy) defines a mapping from
%%%	every possible realization of activity durations to a corresponding realized schedule.
%%%	PS-RCPSP asks to find this tuple that minimizes 
%%%	the \emph{weighted sum} of two performance criteria:
%%%	\begin{enumerate}
%%%		\item expected project makespan,
%%%		\item expected tardiness with respect to the proactive schedule.
%%%	\end{enumerate}
%%%	The first criterion is relevant for obvious reasons.
%%%	The second criterion represents the extent to which the proactive start-times can be trusted.
%%% 	One typically wants to achieve low instability to avoid what is
%%% 	known as \emph{nervousness} \cite{steele1975nervous} and induces organizational costs.
 	
 	% -- paper structure and contributions
 	
	%This paper is about a problem of practical value in project scheduling:
	%trading expected makespan for stability under stochastic uncertainty.
	
	The contributions of this paper extend from section~\ref{sec-problem} and onwards.
	In order to base the paper on a consistent and self-contained framework of notation,
	section~\ref{sec-prelim} summarizes existing concepts from
	deterministic, reactive, and proactive-reactive project scheduling.
 	Section~\ref{sec-problem} introduces in a formal manner the problem studied here,
 	that we name Proactive Stochastic RCPSP.
	Section~\ref{sec-lp} presents one main contribution of this paper:
	a LP-based heuristic for solving this problem.
 	Section~\ref{sec-milp} presents another main contribution:
 	a MILP model for this problem which enables us to obtain exact PS-RCPSP solutions.
 	To our knowledge, no other exact solution methods have been proposed for poblems of similar type.
	Section~\ref{sec-milp-heuristic} presents yet another contribution,
	a MILP-based heuristic for PS-RCPSP.
	Section~\ref{sec-experiments} presents an experimental study in which we find
	that the LP-based heuristic performs favorably in comparison to the state-of-art,
	especially when the aim is to achieve near-zero instability.
	We also find the MILP-based heuristic to be more effective (albeit less efficient)
	when one is willing to accept medium levels of instability in order to minimize expected makespan.
	Section~\ref{sec-conclusion} concludes the paper.	
 	
