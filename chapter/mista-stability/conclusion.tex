\section{Conclusions and future work}
\label{sec-conclusion}

This paper proposes the PS-RCPSP problem model which,
assuming stochastic activity durations,
asks to find a so-called earliest-start (es) policy and
a proactive schedule that together minimize the weighted 
sum of expected project makespan and expected instability.
Extending an existing MILP model for the RCPSP,
a MILP model for PS-RCPSP is presented, which allows us to 
find optimal (es-policy, proactive schedule) pairs.
Solving this problem to optimality might require an impractical amount of time,
even for instances with few activities (e.g. 30).
Therefore, we propose a LP-based and a MILP-based heuristic for the PS-RCPSP.
Our LP-based heuristic optimizes the proactive schedule by keeping the es-policy part of the  solution fixed.
Our MILP-based heuristic optimizes the structure of the policy together with the proactive schedule.
The LP-based heuristic, which is rather efficient, seems to be more effective compared to the state-of-art
(i.e. achieves smaller expected makespan for a certain level of expected instability) especially
when the aim is to achieve close to zero instability.
The MILP-based heuristic is rather effective when the aim is to
achieve low expected makespan at the cost of moderate or high instability.
In contrast to existing state-of-art approaches such as
CCP \cite{lamas2015} and STC \cite{van2008},
our heuristics rely on the idea of optimizing 
the proactive schedule together with the scheduling policy.
This difference might in part explain observed performance differences.

Future work involves a thorough experimental analysis of the proposed heuristics,
not for the purpose of comparing them to the state-of-art,
but for a deeper understanding of their behavior and its
dependence on problem characteristics.
Furthermore, most existing stochastic project scheduling works are evaluated on instances where
the deterministic RCPSP counterpart instance 
(formed by mean activity durations) serves as a good approximation of the stochastic instance.
This is exploited by our heuristics and other heuristics such as STC.
However, in certain practical domains (e.g. maintenance scheduling),
the duration of some activities is known a-priori with accuracy,
while the duration of other activities follows a distribution with very high variance.
In maintenance scheduling, for example, "inspection" activities have known durations
but "repair" activities might be (un)necessary with certain probabilities. 
We would like to investigate performance on such instances 
which cannot be approximated well by their determinitic counterpart.