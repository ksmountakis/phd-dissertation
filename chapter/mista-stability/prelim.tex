\section{Preliminaries}
	\label{sec-prelim}
 	
 	The purpose of this section is to introduce the research area of
 	proactive-reactive (stochastic) project scheduling,
 	which is where the contributions of this paper belong to.
  	To establish autonomy and to facilitate discussion in further sections,
  	we use convenient notation (which sometimes departs from standard notation)
  	and begin with summarizing existing concepts from 
  	deterministic and (purely) reactive project scheduling.
  	For a comprehensive survey of deterministic,
  	reactive and proactive-reactive project scheduling,
  	the reader may refer to \cite{herroelen2004robust}.
   	 	
 	\subsection{Deterministic project scheduling}
 	 	
	A project is usually represented as a directed acyclic graph $G(N,E)$,
	with nodes $N=\{1,\ldots,n\}$ corresponding to the set of $n$ project activities.
	Each directed arc $(i,j)$ in $E \subseteq \{(i,j) \in N^2\}$ defines a direct
	temporal constraint between activities $i$ and $j$,
	meaning that $j$ may not start unless activity $i$ has finished.
 	In effect, $E$ defines a binary, irreflexive and transitive relation:
 	if there is a path from activity $i$ to activity 
 	$j$ in $G(N,E)$ then $j$ cannot start unless $i$ has finished.
	Let us $T(E) \supseteq E$ denote the transitive closure of $E$, 
	defined as 
	\[
		T(E) := \{(i,j) \in N^2 : \exists 
		\textrm{ a path from } i \textrm{ to } j \textrm { in } G(N,E)\}
	\]
	We shall name a \emph{temporally independent set} each
	subset of activities $X \subseteq N$ which are mutually 
	independent with respect to temporal constraints.
	That is, if $X$ is a temporally independent set, then $X^2 \cap T(E) = \emptyset$.
	Obviously, if only temporal constraints are taken into account, 
	the activities of a temporally independent set may overlap in time in a schedule.
	
	We assume as input a set $R := \{1,\ldots,m\}$ of $m$ resources which must be shared among activities.
	Each resource $r \in R$ is associated with known capacity $b_r \in \mathbb{N}_0$.
	Furthermore, each activity $i \in N$ requires a known amount 
	$q_{ir} \leq b_r$ of resource $r$ while it executes.
	Vector $\xb \in \mathbb{N}^m_0$ and matrix $\xq \in \mathbb{N}^{n \times m}_0$
	define the problem's resource constraints.
	Every independent set $X$ for which $\sum_{i\in X} q_{ir} > b_r$ 
	for some $r\in R$ is called a \emph{forbidden} set.
	Even though it is allowed by the temporal constraints $E$, 
	all activities in $X$ may not overlap at some timepoint $t$ because
	resource $r$ will be used beyond its capacity, which is not possible.
	
	Let $H \subseteq N^2$ denote a set of temporal constraints.
	Below we give the definition of a function $\Phi$ which returns the set of all forbidden sets
	w.r.t. temporal constraints $H$ and the problem's resource constraints $(\xq,\xb)$.
	\begin{align}
		\Phi(H) := \{X \subseteq N: 
		X^2 \cap T(H)=\emptyset, 
		\sum_{i\in X} q_{ir} > b_r \textrm{ for some } r \in R\}
	\end{align}
	
	In addition to the parameters mentioned so far,
	we assume as input a vector $\xd \in \mathbb{N}^n_0$ such that $d_i$ defines the duration of activity $i$.
	Overall, a tuple $(N,R,E,\xd,\xq,\xb)$ specifies an instance of the RCPSP.
	A schedule $\xs \in \mathbb{N}^n_0$ 
	such that $s_i$ defines the start time of activity $i$,
	is a feasible solution when it satisfies the temporal and resource constraints,
	meaning that
	\begin{align}
		s_j \geq s_i + d_i \qquad & \forall (i,j) \in E \label{eq-precon} \\
		a(\xs, t) \notin \Phi(E)	\label{eq-rescon} & \qquad \forall t \geq 0
	\end{align}
 
	Here, $a(\xs,t) := \{i \in N : t \in [s_i, s_i+d_i)\}$ is the set of
	activities executing at timepoint $t$ according to $\xs$ and $\Phi$ as defined earlier.
	Thus, (\ref{eq-rescon}) ensures there is no timepoint $t$ at which the
	activities of a forbidden set overlap concurrently.
	
	\paragraph{Project source-sink convention.}
	RCPSP asks to find a feasible schedule of minimum makespan
	$C_{\max}(\xs) := \max \{ s_i + d_i : i \in N\}$.
	Most RCPSP-related works assume that the last activity, $n$, 
	is a dummy activity with zero duration (i.e. $d_n = 0$) and that
	it must wait for the completion of every other activity (i.e. $(i,n) \in T(E)$ for every $i \in N-\{n\}$).
	This dummy activity is often known as the project "sink" and it holds that $C_{\max}(\xs) = s_n$.
	Another convention of most RCPSP-related works is that the first activity, 1,
	often known as the project "source" must be waited on by every other activity
	(i.e. $(1,j) \in T(E)$ for every $j \in N-\{1\}$).
	
 	We shall hereafter assume activities 1 and $n$ correspond to the project source and sink, respectively.
 	The RCPSP can now be formally stated as:
	\begin{align}
		\xs^* := \arg \min \{s_n:  (\ref{eq-precon}), (\ref{eq-rescon}), \xs \geq 0 \}
	\end{align}
	
	\subsection{Reactive project scheduling}
 	
	In the research area of stochastic project scheduling,
	the activity durations vector $\xd$ is replaced with a stochastic vector $\xD$ such that $D_i$
	is the stochastic variable representing the uncertain duration of activity $i$,
	with a known probability distribution $\mathbb{P}[D_i = t]$.
	In line with recent works on S-RCPSP,
	we shall denote (elements of) stochastic vectors with a capital symbol.
			
	S-RCPSP is a purely reactive extension of RCPSP.
	The solution sought for is no longer a schedule, but a reactive scheduling policy. 
	A policy is a combinatorial object $\pi$ which parameterizes the mapping  from stochatic vector $\xD$
	to a corresponding realized schedule $\xS(\pi,\xD)$.
	Note that $\xS$ denotes a function which returns a vector of activity start times (of length $n$).
	Furthermore, if a realization $\xd$ of $\xD$ is passed as an argument, 
	then $\xS(\pi,\xd)$ denotes a deterministic vector.
	if $\xD$ is passed as an argument, $\xS(\pi,\xD)$ denotes a stochatic vector.
	
	Different classes of policies have been proposed in the literature
	\cite{mohring1984stochastic,mohring1985stochastic,stork2000branch,ashtiani2011new}.
 	One condition that all policy classes are expected to satisfy is that function
 	$\xS$ complies with the \emph{non-anticipativity constraint}:
	the decision to start activity $i$ at time $[\xS(\pi,\xD)]_i$ cannot rely on information from the feature:
	the value of $[\xS(\pi,\xD)]_i$ must be determined by time $t \leq [\xS(\pi,\xD)]_i$.
	Other features such as the structure of $\pi$ and 
	the definition of function $\xS$ depend on the class under study.
	
 	\paragraph{List-based policies.}
	
	Two classes of policies which are prominent in the literature
	are \emph{resource-based} (rb) policies and \emph{activity-based} (ab) policies,
	also known collectively as \emph{list-based policies}.
	A list-based policy is a priority vector $\xl \in \mathbb{R}^n$ assigning priority $l_i$ to activity $i$.
	Realized schedule $\xS(\xl, \xD)$ is computed by a 
	variant of the well-known parallel schedule-generation-scheme (SGS)
	complying with the non-anticipativity constraint \cite{ballestin2009resource};
	with the SGS definition being slightly different between rb-policies and ab-policies.
%	At $t=0$ and at each subsequent activity completion $t > 0$
%	one starts as many activities as allowed by temporal and resource constraints.
%	A similar class is that of \emph{activity-based} (ab) policies;
%	an ab-policy is again a priority vector.
%	But the behavior of $p$ is now slightly different:
%	a activity may not start at $t$ 
%	(even if temporal and resource constraints allow this) unless 
%	all activities of lower priority have finished by $t$.
%	Note that every realized schedule $\xS(\xl,\xD)$ is feasible, 
% 	since function $\xS$ forces the problem's precedence and resource constraints
% 	regardless of the choice of $\xl$.
 	As far as list-based policies are concerned, S-RCPSP asks to find a vector 
	$\xl \in \mathbb{R}^n$ that minimizes $\mathbb{E}[[\xS(\xl,\xD)]_n]$.
	%
	Stork \cite{stork2000branch} proposes exact branch-and-bound algorithms for both rb and ab-policies.
	Ballest{\'\i}n \cite{ballestin2007worthwhile} proposes
	an efficient genetic algorithm for ab-policies,
 	providing the first computational experience on larger S-RCPSP instances.
	Ballest{\'\i}n and Leus \cite{ballestin2009resource} manage 
	to obtain better results with a 
	Greedy Randomized Adaptive Search Procedure (GRASP), 
	again for the class of ab-policies.
	The best performance (w.r.t. expected makespan minimization)
	has so far been obtained with the more recent work of
	Ashtiani et al. \cite{ashtiani2011new} who propose a GRASP
	for a new class, namely \emph{pre-processing} (pp) policies--%
	a hybrid between rb-policies and es-policies.

 	\paragraph{Earliest-start policies.}
 	
   	An es-policy is a set of temporal constraints $\xE \subseteq N^2$ chosen such that
 	\begin{align}
 		T(\xE) \supseteq E, \label{es-1} \\
 		\Phi(\xE) = \emptyset \label{es-2}, \\
 		G(N,\xE) \textrm{ is acyclic} \label{es-3}
 	\end{align}
 	Condition (\ref{es-1}) ensures that a schedule $\xs$ satisfying
 	$s_j \geq s_i + d_i$ for each $(i,j)\in \xE$ 
 	(here $\xd$ can be any arbitrary choice of activity durations)
 	is feasible with respect to the problem's precedence constraints $E$.
  	Condition (\ref{es-2}) ensures that
  	$\xs$ satisfying $\xE$ implies that it also satisfies resource constraints 
  	prescribed by availabilities $\xb$ and requirements $\xq$ (as described earlier).
 	Condition (\ref{es-3}) ensures that the set of schedules satisfying $\xE$ 
 	(for any arbitrary choice of activity durations $\xd$) is non-empty.
 	
 	When a project is executed according to an es-policy $\xE$,
 	the schedule that is realized, $\xS(\xE,\xD)$, 
 	is what is often known as the \emph{earliest-start} schedule specified by $\xE$.
 	The earliest-start schedule of $\xE$ can be defined as:
 	\begin{align}
 		[\xS(\xE,\xD)]_j := \max \{ [\xS(\xE,\xD)]_i + D_i : (i,j) \in \xE\}
 	\end{align}
 	To put it simply, an activity $j$ starts immediatelly 
 	when all its predecessors in $G(N,\xE)$ have finished.
  	This time quantity (the latest finish time of $j$'s predecessors)
  	is often known as the length of the \emph{critical path} from project source $1$ to activity $j$.
  	As far as es-policies are concerned, the S-RCPSP asks to find some $\xE^*$ which
  	minimizes $[\xS(\xE,\xD)]_n$ (the length of the critical path to the project sink) by expectation:
 	\begin{align}
 		\xE^* := \arg \min \{\mathbb{E}[[\xS(\xE,\xD)]_n] : (\ref{es-1} - \ref{es-3}), \xE \in N^2\}
 		\label{eq-s-rcpsp}
 	\end{align}
  	Constraints $(\ref{es-1} - \ref{es-3})$ enforce the feasibility of any realized schedule
  	with both the precedence and the resource constraints of the problem.
  	
	Stork \cite{stork2000branch} proposed an exact branch-and-bound search for problem (\ref{eq-s-rcpsp}).
	His algorithm considers each \emph{minimal} forbidden set $X$
	(subset-minimal forbidden set) in some order and branches
	on each of $|\{(i,j) \in X^2\}|$ arcs which can be included in $\xE$
	in order to eliminate $X$ from $\Phi(\xE)$.
	Without obtaining new computational results,
	in \cite{leus2011resource} Leus gives a formal treatment of
	es-policies as resource-flow networks
	(flow networks which can represent feasible RCPSP schedules) and
	proposes a refined version of the branch \& bound algorithm of Stork.
	Exploiting the relation between resource-flows and es-policies,
	Artigues et al. \cite{leus2011robust} propose a 
	robust optimization model for es-policies,
	for when a stochastic characterization of uncertainty is not available.
	
\subsection{Proactive-reactive project scheduling}
	
	Reactive project scheduling allows one to pick activity start times dynamically
	during the project, under conditions of uncertainty.
	A main drawback of this approach 
	(e.g. \cite{herroelen2004construction,herroelen2004robust,braeckmans2005proactive})
	is that prior to (and during) project execution there is no 
	schedule prescribing activity start times that can more or less be trusted.
	Such a "proactive" schedule can serve important organizational purposes;
	in fact, the deviation of the realized schedule from this proactive schedule
	is expected to induce organizational costs.
	
	Attempts to overcome this drawback gave rise to the research area of
	proactive-reactive project scheduling,
	which is the research area that this paper belongs to.
	The main idea behind the proactive-reactive approach is to execute the project by using
	a proactive schedule together with a scheduling policy.
	Under uncertainty, some activities may not start at their proactive start times,
	because activities they have to wait for are not yet finished and/or
	resources they require are not yet released.
	In such cases, the scheduling policy determines which activities to start at their
	prescribed start times and which to postpone.
	It should be noted that most works assume \emph{railway-mode scheduling},
	meaning that an activity may not start earlier than its proactive start time,
	which strengthens the "stability" of the project execution.
	Clearly, the realized schedule is a function of the policy and the proactive schedule.	
	Achieving low instability (deviation of the realized from the proactive schedule)
	requires "spreading" proactive activity start times far appart, 
	in effect increasing the expected makespan.
	The general aim is to optimize some tradeoff between expected makespan and expected instability.
	
	Van de Vonder et al. \cite{van2006trade,van2008,van2007heuristic} propose and
	evaluate experimentally various proactive-reactive heuristics.
	The proposed heuristics assume as input an instance of S-RCPSP along with an initial schedule.
	The best performing heuristic is the so-called Starting Time Criticality (STC) heuristic.
	An es-policy is extracted from the structure of this initial schedule and used to
	iteratively transform the initial schedule into a 
	proactive schedule by inserting time-buffers betwen activities.
	%
	Deblaere et al. \cite{deblaere2011proactive} propose 
	an approach which integrates the proactive step (forming a proactive schedule) 
	and the reactive step (forming the adjoining policy).
	Their approach is only possible to compare with ours and others that assume railway-mode scheduling,
	by choosing sufficiently high penalties for earliness (w.r.t. the proactive schedule).%
	\footnote{We are grateful to one of our anonymous reviewers for this remark.}
 	%
	More recently, Vilches and Demeulemeester  \cite{lamas2015}
	propose a Chance-Constrained Programming model (CCP) for the RCPSP
	which asks to find a minimum makespan schedule
	subject to probabilistic temporal and resource constraints.
	They propose a Mixed Integer LP model, 
	the solution to which is a proactive schedule that will most likely remain 
	feasible under stochastic duration variability,
	without presumption on the policy that will be used during project execution.
