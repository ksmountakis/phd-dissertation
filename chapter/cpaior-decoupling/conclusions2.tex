\section{Conclusions and Discussion}

We have shown that adapting a decoupling after a new variable commitment has been
made in most cases significantly increases the flexibility of the free, non-committed, variables.
We also showed that this updating method did not induce any disadvantage to the actors involved: every commitment
is preserved and the current decoupling bounds are never violated.
%To obtain such a decoupling update method, we used an exact method based on computing minimum matchings.
We introduced a simple heuristic that replaces the rather costly computation of a decoupling with maximum flexibility by a computation of a decoupling with maximal flexibility. This heuristic reduces the computation time per update from $O(n^3)$ to $O(n.\log n)$ (amortized) time.
As we showed experimentally, the computational investments for the heuristic are significantly smaller, while we observed almost no loss of flexibility compared to the exact method.

In the future, we want to extend this work to computing flexible schedules for STPs: Note that the update heuristic 
also can be used to find a maximal flexible schedule given a solution $s$ of an STP. Such a solution is nothing more than
a non-optimal decoupling $(s,s)$ for which, by applying our heuristic, an $O(n \log n)$ procedure exists to find an optimal
flexible schedule. Furthermore, we are planning to construct dynamic decoupling methods in a distributed way, like existing approaches to static decoupling methods have done.