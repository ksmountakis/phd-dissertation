%
% -- a first attempt at defining a plan for this chapter
%
%This is probably the trickiest chapter of the thesis as it needs to ``bring under a common roof'' the
%two very different topics addressed in this thesis: simple temporal problems on one hand and stochastic task networks on the other.
%What follows outlines my vision for this chapter.
%
%\subsubsection*{Scheduling problems and their applications}
%Introduce the reader to scheduling problems with a high-level summary of the diferent types of problems and their respective real-world applications.
%A non-technical, high-level outline of different types of constraints, parameters and objectives.
%
%A high-level definition of the concept of a schedule,
%as an artifact that prescribes the placement of events in time, subject to the problem constraints.
%Such events can be instantaneous or not, depending on the problem.
%
%\subsubsection*{Uncertainty in scheduling}
%Introduce the reader to the concept of scheduling under uncertainty. 
%Why taking into account uncertainty is necessary in most (if not all) real-world applications of scheduling.
%(Perhaps here we can make a ``first-pass'' on the link with NedTrain's rolling-stock logistics programme.)
%
%\subsubsection*{High-level look at the two parts the thesis and the underlying theme}
%This thesis summarizes results for two different types of scheduling problems and as such it is divided into respective parts I and II.
%Both types of problems belong to the research area of scheduling under uncertainty, but uncertainty has a different meaning in each of the two cases.
%Despite their differences, parts I and II have an underlying theme in common: 
%beyond finding a fixed schedule for a given scheduling problem,
%we are interested in finding a strategy for adjusting the solution in a continuous scheduling process.
%
%\subsubsection*{More details on Parts I and II}
%Part I focuses on problems related to dispatching instantaneous events subject to Simple Temporal Problem (STP) constraints.
%Such constraints restrict the minimum and maximum temporal distance between the dispatching of pairs of events.
%It is assumed that each event is associated with a respective actor that will choose when to dispatch his event from a respective time-window.
%Uncertainty in this case results from not being able to predict how an actor will choose within a given time-window.
%We are interested in finding a \emph{flexible} strategy for initializing these time-windows and updating them as choices are being made,
%striving to maximize the freedom with which actors can make choices as events keep getting dispatched.
%
%Part II focuses on problems related to dispatching non-instantaneous events that correspond to tasks with random durations,
%subject to simple precedence constraints and resource constraints.
%Uncertainty in this case results from not being able to predict the outcome duration of a task.
%We are interested in finding a dispatching strategy that enables us to predict with some confidence the outcome start-times of the tasks, 
%even though the outcome task durations are unpredictable.
%More in particular, we are interested in finding a strategy that optimizes the trade-off between two conflicting qualities:
%dispatching tasks efficiently and dispatching tasks predictably.
%
%\subsubsection*{The connection with NedTrain}
%\ldots
%
%\subsection*{Thesis structure}
%\ldots
%

STPs allow us to
- model constraints in certain AI problems,
- compactly represent a space of feasible schedules.

A feasible schedule can be constructed by alternating 
the steps of fixing variables to values, and updating the minimal STN.

An all-pairs-shortest path algorithm allows us to associate variables with respective intervals.
Such intervals have the property that if some variables are fixed to certain values,
then it is always possible to extend this partial assignment to a feasible schedule for all variables.


Each time a subset of the variables are fixed to certain values,
consistency algorithms allow us to:
- efficiently update this space of feasible schedules,
- efficiently verify if the resulting STP remains consistent.

What does Hunsberger's research help us do?
Same for Boerkoel's

1. Check Michel Wilson's thesis and related papers for literature review.
1a. Download two of his papers: done
1b. Search in Krishna's thesis:
1b1. Read Chapter 1:
1c. Search in Michel's thesis:

