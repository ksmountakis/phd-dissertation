Het onderzoek dat gepresenteerd wordt in dit proefschrift is onderdeel van Rolling Stock Life Cycle Logistics (Rollend Materieel Levenscycluslogistiek), 
een toegepast onderzoeks en ontwikkelingsprogramma, uitgevoerd door NedTrain. 
Nedtrain is een bedrijf dat behoort tot de Nederlandse Spoorwegen (NS, het voornaamste spoorwegbedrijf in Nederland) 
en biedt onderhoudsdiensten aan de treinvloot van de NS. 
Het doel van dit programma is om het concurrentievermogen van NedTrain te verbeteren als leverancier onderhoud service voor rollend materieel. 
Ons werk richt zich op het operationele aspect van dit R\&D-programma, 
gemotiveerd door de uitdaging van het plannen van taken (of werkzaamheden) in een onderhouds\-werkplaats van NedTrain, 
zodat treinen op tijd worden afgeleverd voor circulatie in het spoorwegnet. 
De meeste taken die in de werkplaats plaatsvinden hebben een onzekerheid in de duur (of verwerkingstijden), 
hetgeen het plannings\-proces bemoeilijkt.

Na de introductie van NedTrain als bedrijf, 
worden in Hoofdstuk 1 de belangrijkste problemen geïdentificeerd met betrekking tot plannen in de werkplaats. 
Het punt wordt gemaakt dat plannen onder onzekerheid niet zozeer gaat om het vinden van een goede (of tijdige) planning, 
maar om het voortdurend aanpassen van de planning aan de resulterende looptijden van de taken zonder de gestelde beperkingen 
in de planning te overtreden en zonder afbreuk te doen aan tijdigheid. 
De planning van de werkplaats herhaaldelijk wijzigen is geen optie omdat dat personeel zou verwarren en desorganiseren, 
en daarmee de prestaties belemmeren. Als zodanig beschouwen we twee mogelijkheden waar het management heeft om met de onzekerheid om te gaan: 
i) in plaats van een planning te gebruiken die vaak verandert, mensen de flexibiliteit te geven om zichzelf te (her)plannen naar believen; 
ii) voldoende speling in de planning toe te voegen om frequente wijzigingen tijdens de uitvoering van de taak te voorkomen. 
Elke bovenstaande optie is toegewezen aan een overeenkomstig onderzoeksprobleem dat hieronder is weer\-gegeven: 

\begin{enumerate}[(I)]
\item{Hoe flexibele plannen te berekenen voor onafhankelijke werkteams, plannen die gemakkelijk kunnen worden aangepast aan de veranderingen in de omgeving?}
\item{Hoe robuuste en stabiele planningen te berekenen voor werkteams om te kunnen omgaan met onzekerheid in de duur van onderhoudstaken?}
\end{enumerate}

Bij het nastreven van optie I hierboven, 
is het doel om zoveel mogelijk flexibiliteit te bieden aan onafhankelijke besluitvormende partijen in een werkplaats. 
De grootste moeilijkheid is ervoor te zorgen dat beperkingen op de planning 
(d.w.z. voorrang tussen taken, opleveringsdata, beperkingen op de beschikbaarheid van hulpbronnen) 
voldaan worden door een plan dat geleidelijk is gevormd door besli\-ssingen die onafhankelijk van elkaar worden genomen door verschillende partijen. 
Het doel van het nastreven van optie II, aan de andere kant, 
is om zoveel stabiliteit te bieden (d.w.z. een plan die naar verwachting niet zal veranderen) als mogelijk zonder afbreuk te doen aan prestaties (d.w.z. tijdige levering van treinen). 
De moeilijkheid in het nastreven van optie II is het bepalen van de hoeveelheid speling om in te voegen en op welke momenten deze in te voegen in het plan, 
afhankelijk van hoe onzekerheid accumuleert op de verschillende delen van het plan.

Onderzoeksprobleem I wordt behandeld in Deel I van het proefschrift. 
In de traditie van het vroegere werk in het kader van RSLCL, 
bekijken we op het onderzoeksgebied van Simple Temporal Problem (STP) beperkingen. 
Het belangrijkste idee achter onze aanpak is het modelleren van de situatie in een NedTrain-werkplaats (d.w.z. temporele beperkingen en beperkingen op de hulpbronnen) als een STP. 
Dit STP is toegewezen aan een intervalplan. 
In tegenstelling tot een regulier plan (dat een verzendtijd per taak voorschrijft), 
schrijft een intervalplan een tijdsinterval per taak voor. Zolang elke taak binnen zijn tijdsinterval wordt verzonden, 
is de voldoening aan de beperkingen in de werkplaats gegarandeerd.

Deel I is toegewijd aan argoritmen die een intervalplan van maximale flexibiliteit vinden (ze voorstellen zo breed mogelijk tijdsintervallen) 
en die dit plan bijwerken met de nieuwe informatie over de al verzonden taken naarmate deze informatie beschikbaar wordt tijdens uitvoering. 
Hoofdstuk 2 is een soort voorspel voor Deel I. 
Het hoofdstuk begins met een samenvatting van belangrijke concepten van de de STP-gerelateerde literatuur. 
Daarna richt het zich op het identificeren van de literetuugaten die ons niet toestaan om Onderzoeksprobleem I aan te pakken. 
Eindelijk formuleren we een reeks onderzoeksvragen die met de gaten te maken hebben, 
en we bekijken deze vragen in Hoofdstukken 3 en 4. 
De gaten in de bestaande literatuur komen in wezen voort uit de effictiviteit en dynamicity opzichten die niet bestudeerd werden in het vorige werk door Wilson et al. 
Hier stellen wij de volgende onderzoeksvragen:
\begin{enumerate}
	\item{Hoe kan men gelijktijdige flexibiliteit effeciënt berekenen in een gegeven STP, binnen een lage-orde polynomische tijd?}
	\item{Hoe kan men stapsgewijs een intervalplan van gelijktijdige flexibiliteit herberekenen tijdens het verzenden?}
	\item{Hoe kan men de gelijktijdige flexibiliteit zo snel mogelijk opnieuw verzenden (door middel van heuristieken, indien nodig)?}
\end{enumerate}

Wilson et al. hebben een methode bedacht om de interval plannen van maximum gelijktijdige flexibiliteit te berekenen, 
en deze methode neemt $O(n^5)$ tijd, waar n is het aantal taken. 
In het gevan van NedTrein, kan n in de orde van duizenden zijn, 
en daarom gaat de eerste vraag erboven over de ontwikkeling van een meer efficiënte intervalplannenmethode. 
Onderzoeksvragen 2 en 3, aan de andere kant, gaan over de toevoeging van een dynamische dimensie tot het statische gelijktijdige flexibiliteitskader, 
dat oorsproonkelijk door Wilson et al. geontwikkeld is. 
Deze vragen dus bekijken op de ontwikkeling van methoden voor het bijwerken van het intervalplan naarmate de nieuwe informatie over de al verzonden taken bekend wordt.

Hoofdstuk 3 en het eerste deel van Hoofdstuk 4 bestuderen de eerste vraag. 
Door middel van een geometrische interpretatie van flexibiliteit en met behulp van de dualiteitstheorie, 
demonstreren we dat de berekening van flexibiliteit kan worden geworpen als het vinden van een min-kost koppeling in een gewogen bipartiete graaf. 
Dit geeft ons de mogelijkheid om de flexibiliteit te berekenen binnen $O(n^3)$ door gebruik van een min-kost koppelingsalgoritme (bijvoorbeeld, de Hongaarse methode). 
Dat verbetert de $O(n^5)$ grens van de aanpak die gebaseerd is op Lineair Programmeren (LP), een aanpak die oorsproonkelijk Wilson et al. hebben voorgesteld.

Hoofdstuk 4 dan gaat door en nadert vragen 2 en 3. 
Wat vraag 2 betreft, er wordt laten zien hoe, gegeven een intervalplan, een nieuw intervalplan berekend kan worden elke keer dat een taak verzonden wordt op een punt binnen het betreffende tijdsinterval. 
De flexibiliteit beschickbaar in het tijdsinterval is niet meer nodig nadat de taak verzonden wordt. 
Daarom, bestuderen we het probleem van herverdeling van de niet verbruikte flexibiliteit naar de nog-niet-verzonden taken. 
Dit laat ons de flexibiliteit per-taak voortdurend verhogen, 
terwijl de taaksuitvoering zich ontvouwt. Om geen verwarring te veroorzaken door het voortdurend veranderende intervalplan, 
is elke verzendoptie die beschikbaar is in een gegeven plan ook in het bijgewerkte plan beschikbaar. 
Dat wilt zeggen, bevatten de tijdsintervallen van het bijgewerkte plan de tijdsintervallen van het gegeven plan. 
Het opnieuw verdelen van flexibiliteit (of het plan bijwerken) is eigenlijk een herplanningsoperatie, 
en daarom zou het berekenbaar zo efficiënt mogelijk zijn, 
om gelijke tred te houden met de uitvoering. 
Het laatste deel van Hoofdstuk 4 gaat over de ontwikkeling van een zeer efficiënte heuristiek.

Onderzoeksprobleem II wordt behandeld in Deel II van het proefschrift. 
Hier is het belangrijkste idee de situatie in de werkplaats modelleren (d.w.z. de temporale en hulpbronbeperkingen) als een stochastische taaknetwerk, 
d.w.z. een netwerk van voorrangrelaties tussen de taken met randomise duurtijden. 
Onze belangrijkste aanname is dat we de kansverdeling van elke taaksduur kunnen inschatten door middel van historische data van de onderhoudssessies van het verleden. 
Met gebruik van dit precieze beeld van onzekerheid in de werkplaats, 
wordt het taaksnetwerk toegewezen aan een voorspellend (of stabiel) plan. 
De mensen in de werkplaats worden gevraagd om de taken eenvoudig te verzenden op de verzendtijden die in dat plan staan. 
Het voorspellende plan wordt gemaakt met gebruik van de juiste hoeveelheid speling op de juiste plekken, 
zodat dat grotendeels hetzelfde blijft gedurende de uitvoering van de taken.

Deel II is gewijd aan algoritmen die zo’n stabiel voorspellend plan vinden zonder afbreuk te doen aan robuustheid (d.w.z. de kans van op tijd zijn met de treinaf\-leveringsdata) 
en (optioneel) die dat bijwerken naarmate nieuwe informatie komt gedurende de uitvoering van een taak.  
Hoofdstuk 5 speelt de rol van een voorspel voor Deel II. 
Het begint met een samenvatting van de belangrijke concepten van de literatuur over Stochastische TaakNetwerken (Stochastic Task Networks). 
We identificeren de gaten in de literatuur die het bekijken van Onderzoeksprobleem II onmogelijk maken, 
en dat brengt ons tot het formuleren van een reeks onderzoeksvragen die met deze gaten te maken hebben. 
Deze vragen zijn bestudeerd in Hoofdstukken 6 en 7. Hier stellen we de volgende twee onderzoeksvragen:
\begin{enumerate}
	\item{Hoe kan men een plannigsbeleid en een voorspellend plan samen optimaliseren als een paar?}
	\item{Hoe kan men het voorspellende plan bijwerken door op de resulterende looptijden te reageren binnen lage-orde polynomische tijd, terwijl we gelijke tred met de uitvoering houden.}
\end{enumerate}
Zelfs als het voorspellende plan beschermd is door het invoegen van speling, 
kan het gebeuren da een taak langer duurt dan gepland. 
Een verzameling regels die bepalen hoe het plan veranderd moet worden (als het moet) in zulke gevallen is bekend als een planningsbeleid. 
Eigenlijk, wordt het proces van taakuitvoeren geleid door het voorspellende plan samen met het planningsbeleid. 
De vooraan\-staande aanpakken in de literatuur werken afzondelijk aan het optimeseren van het beleid en het plan. 
In het nastreven van mogelijk betere resultaten door te optimeseren in een oplossingsruimte van een hogere dimensie, 
gaat de eerste vraag over hoe het plan en het beleid samen geoptimaliseerd kunnen worden, als een oplossing. 
Als we al hebben gemerkt, kunnen de tijden van het resultaatsverzenden afwijken van degenen in het voorspellende plan. 
Als dit het geval is, kan het volgende nuttig zijn: 
de spelingstoewijzing aanpassen volgens de verzendtijden van het resultaat en de looptijden van de taken. 
De tweede vraag gaat over het voorspellende plan bijwerkend houden naarmate zo’n nieuwe informatie beschikbaar wordt  tijdens de uitvoering. 
Omdat dit een herplanningsoperatie is, concentreren we op deze zo efficiënt mogelijk uit te voeren.

De eerste vraag hierboven wordt in Hoofdstuk 6 behandeld, 
waar we een aanpak van twee stappen (het plan en het beleid afzondelijk optimaliseren, in twee stappen) 
en een geïntegreerde aanpak (het beleid en het plan samen optimaliseren) ontwikkelen. 
Dat doen we door een stochastische extensie van de bestaande MILP formuleringen te ontwikkelen voor het (deterministische) 
Resource Constrained Project Scheduling Problem (RCPSP). 
Zoals verwacht, levert de geïntegreerde aanpak betere resultaten. 
Helaas, is het te rekenkundig duur voor problemen van praktische grootte. 
Aan de andere kant, lijkt onze twee stappen aanpak beiden efficient en heel effectief te zijn in het optimaliseren van het voorspellende plan voor een gegeven planningsbeleid; 
deze aanpak oplevert betere resultaten dan de state of the art.

De tweede vraag hierboven wordt in Hoofdstuk 7 behandeld. 
We stellen eerst een LP voor voor het opbouwen van een voorspellend plan, 
gebaseerd op een steekproef van randomise taaklooptijden. 
Dit LP is rekenkundig duur om op te lossen, met een complexiteit van $O(n^5 m^4)$, 
waar n het aantal taken betekend en m de grootte van de steekproef van de looptijden is. 
Met het oog op een bepaald deel van de oplossingsruimte, 
definiëren wij het erbij horende Simple Temporal Problem (STP) 
en we laten zien dat het optimale voorspellende plan gemaakt kan worden van de vroegste-begin-tijd oplossing van het STP. 
Door gebruik te maken van de speciale STP structuur, 
presenteren wij onze belangrijkste resultaat, 
een dynamisch programmeren algoritme dat een optimaal voorspellend plan vindt, 
een plan dat $O(n^2 m)$ kost, en dat aanzienlijke efficiëntiewinsten oplevert.

Hoofdstuk 8 rondt het proefschrift af. 
We beoordelen eerst onze antwoorden op de onderzoeksvragen. 
We gebruiken onze antwoorden om door te gaan en de mate van een geschikte behandeling van Onderzoeksproblemen I en II te discusseren. 
We sluiten het hoofdstuk af met een lijst aanbevelingen voor toekomstig werk.
